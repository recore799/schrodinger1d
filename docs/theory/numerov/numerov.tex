\documentclass[11pt]{article}
\usepackage[utf8]{inputenc}
\usepackage[T1]{fontenc}
\usepackage{amsmath,amssymb,bm,physics}
\usepackage{geometry}
\usepackage{hyperref}
\usepackage[spanish]{babel}
\geometry{a4paper,margin=1in}
\date{}
\title{Método de Numerov}

\begin{document}
\maketitle

\begin{abstract}
En este apartado se presenta el método de Numerov, un esquema de integración
numérica de alta precisión para ecuaciones diferenciales de segundo orden
de la forma \(y''(x) = -g(x)y(x) + s(x)\).
El método se basa en la expansión de Taylor hasta quinto orden y elimina los
términos de derivadas impares, alcanzando un error de truncamiento del orden
\(\mathcal{O}[(\Delta x)^6]\).
Su simplicidad y estabilidad lo hacen especialmente útil para resolver
la ecuación de Schrödinger unidimensional en sistemas como el oscilador armónico
y el átomo de hidrógeno.
\end{abstract}

\section{Introducción}
\label{sec:org58b07f0}

Método numérico desarrollado por el astrónomo ruso Boris Vasilyevich Numerov en la década de 1910. Se basa en desarrollar la expansión de Taylor de una función \(y : I \subset \mathbb{R} \to \mathbb{R}\) alrededor de un punto \(x_0 \in I\), donde \(y\) es solución a la ecuación diferencial de segundo orden

\begin{equation}
\label{eq:numerov-eq}
    y''(x) = - g(x)y(x) + s(x),
\end{equation}

con \(g: I \to \mathbb{R}\) y \(s: I \to \mathbb{R}\) como funciones conocidas, definidas en el mismo intervalo \(I\) que \(y\). Para que la expansión de Taylor sea válida hasta el orden requerido, se exige que \(y\) sea al menos \(C^4\) i.e. cuatro veces diferenciable. Además, \(g\) y \(s\) deben ser funciones suficientemente suaves e.g. continuas, para garantizar la existencia de una solución.

\section{Derivación}

El método de Numerov sigue un esquema de diferencias finitas, por lo que comenzamos con una expansión de Taylor, hasta quinto órden. Para un paso adelante \((x= x_{0} + \Delta x)\) y un paso atrás \((x=x_{0} - \Delta x)\), se tiene

\[ y(x_{0} + \Delta x ) = y(x_{0}) + y'(x_{0})\Delta x + \frac{y''(x_{0})}{2!} \Delta x^2 + \frac{y'''(x_{0})}{3!}\Delta x^3 + \frac{y^{(4)}(x_{0})}{4!}\Delta x^4 + \frac{y^{(5)}(x_0)}{5!} + \order{(\Delta x)^6}, \]

\[ y(x_{0} - \Delta x ) = y(x_{0}) - y'(x_{0})\Delta x + \frac{y''(x_{0})}{2!} \Delta x^2 - \frac{y'''(x_{0})}{3!}\Delta x^3 + \frac{y^{(4)}(x_{0})}{4!}\Delta x^4 - \frac{y^{(5)}(x_0)}{5!} + \order{(\Delta x)^6}. \]

Se define una malla uniforme \(x_n = x_0 + n\Delta x\) y se denota:

\begin{itemize}
\item \(y(x_n) \equiv y_n\) (valores de la funcion en puntos de la malla).
\item \(y(x_n \pm \Delta x) \equiv y_{n \pm 1}\) (puntos adyacentes).
\end{itemize}

Así, las expansiones se reescriben como:

\[ y_{n+1} = y_n + y_n' \Delta x + \frac{1}{2} y_n'' (\Delta x)^2 + \frac{1}{6} y_n''' (\Delta x)^3 + \frac{1}{24} y_n^{(4)} (\Delta x)^4 + \frac{1}{120} y_n^{(5)} (\Delta x)^5 + \order{(\Delta x)^6} \]

\[ y_{n-1} = y_n - y_n' \Delta x + \frac{1}{2} y_n'' (\Delta x)^2 - \frac{1}{6} y_n''' (\Delta x)^3 + \frac{1}{24} y_n^{(4)} (\Delta x)^4 - \frac{1}{120} y_n^{(5)} (\Delta x)^5 + \order{(\Delta x)^6} \]

Al sumar ambos desarrollos se obtiene

\[ y_{n+1} + y_{n-1} = 2y_n + y_n''(\Delta x)^2 + \frac{1}{12} y_n^{(4)} (\Delta x)^4 + \mathcal{O} [(\Delta x)^6] \]

Luego, se puede escribir

\[ z_n \equiv y'' = - g_ny_n + s_n, \]

y aplicamos la expresion obtenida

\[ z_{n+1} + z_{n-1} = 2z_n + z_n'' (\Delta x)^2 + \mathcal{O}[(\Delta x)^4] \]

Esta es una expresion para la segunda derivada desarrollando en Taylor hasta tercer orden, de modo que, ademas, se obtiene una expresion para la cuarta derivada \(y^{(4)}\)

\[ z''_n = \frac{z_{n+1}+z_{n-1}-2z_n}{(\Delta x)^2} + \mathcal{O}[(\Delta x)^2] = y_n^{(4)}. \]

sustituyendo en la expresion

\[ y_{n+1} + y_{n-1} = 2y_n + y_n''(\Delta x)^2 + \frac{1}{12} y_n^{(4)} (\Delta x)^4 + \mathcal{O} [(\Delta x)^6] \]

se obtiene la formula de Numerov

 \begin{align*}
  &y_{n+1} = 2y_n - y_{n-1} + (z_n)(\Delta x)^2 + \frac{1}{12}(z_{n+1}+z_{n-1}-2z_n)(\Delta x)^2 + \order{(\Delta x)^6} \\
  &y_{n+1} =  2y_n -y_{n-1} + (-g_ny_n + s_n) (\Delta x)^2 + \frac{1}{12} [(-g_{n+1}y_{n+1}+s_{n+1}) \\
  & \qq{}\qq{} +(-g_{n-1}y_{n-1}+s_{n-1})-2(-g_ny_n+s_n)](\Delta x)^2 + \order{(\Delta x)^6} \\
  &y_{n+1}\qty[1 + \frac{1}{12}g_{n+1}(\Delta x)^2] = 2y_n \qty[1+\qty(\frac{-g_n}{2}+\frac{g_n}{12})(\Delta x)^2] \\
  & \qq{}\qq{} - y_{n-1}\qty[1+\frac{1}{12}g_{n-1}(\Delta x)^2] + \frac{1}{12}(s_{n+1}+10s_n+s_{n-1})(\Delta x)^2 + \order{(\Delta x)^6} \\
  &y_{n+1}\qty[1 + \frac{1}{12}g_{n+1}(\Delta x)^2] = 2y_n \qty[1-\frac{5}{12}g_n(\Delta x)^2] \\
  & \qq{}\qq{} -y_{n-1}\qty[1+\frac{1}{12}g_{n-1}(\Delta x)^2] + \frac{1}{12}(s_{n+1}+10s_n+s_{n-1})(\Delta x)^2 + \order{(\Delta x)^6}.
\end{align*}

Esta expresion nos permite propagar la solucion \(y_n\) conociendo los primeros dos valores \(y_0\) y \(y_1\). Las ecuaciones que se van a resolver por medio de esta formula tienen \(s(x)=0\). Luego, podemos simplificar la formula por medio de la cantidad

\[ f_n \equiv 1 + \frac{1}{12}g_n(\Delta x)^2, \]

haciendo

\begin{align*}
    f_n - 1 &= \frac{1}{12}g_n(\Delta x)^2 \\
    1-5(f_n -1) &= 1-\frac{5}{12}g_n(\Delta x)^2 \\
    6-5f_n &= 1-\frac{5}{12}g_n(\Delta x)^2 \\
    12-10f_n &= 2\qty(1-\frac{5}{12})g_n(\Delta x)^2
\end{align*}

se reescribe la formula de Numerov

\begin{equation}
\label{eq:numerov}
    \boxed{y_{n+1} = \frac{(12-10f_n)y_n-f_{n-1}y_{n-1}}{f_{n+1}}}
\end{equation}




Este metodo se aplica a los sistemas fisicos que se pueden modelar con una ecuacion diferencial del tipo \ref{eq:numerov-eq}. En particular vamos a resolver la ecuacion de Schrodinger unidimensional independiente del tiempo para el sistema de oscilador armonico y el atomo de hidrogeno, notamos que hay mas ecuaciones interesantes que se pueden resolver como la ecuacion de Poisson.
\end{document}
