% Created 2025-04-20 Sun 17:29
% Intended LaTeX compiler: pdflatex
\documentclass[11pt]{article}
\usepackage[utf8]{inputenc}
\usepackage[T1]{fontenc}
\usepackage{graphicx}
\usepackage{longtable}
\usepackage{wrapfig}
\usepackage{rotating}
\usepackage[normalem]{ulem}
\usepackage{amsmath}
\usepackage{amssymb}
\usepackage{capt-of}
\usepackage{hyperref}
\usepackage{physics}
\usepackage{bm}
\usepackage{geometry}
\geometry{a4paper, margin=1in}
\usepackage{hyperref}
\usepackage{listings}
\author{Rafael Corella, Carlos Felix, Bryan Campa}
\date{\today}
\title{Metodo de Numerov}
\hypersetup{
 pdfauthor={Rafael Corella, Carlos Felix, Bryan Campa},
 pdftitle={Metodo de Numerov},
 pdfkeywords={},
 pdfsubject={},
 pdfcreator={},
 pdflang={English}}
\begin{document}

\maketitle
\tableofcontents

\section{Método de Numerov}
\label{sec:org58b07f0}

Método numérico desarrollado por el astrónomo ruso Boris Vasilyevich Numerov en la década de 1910. Se basa en desarrollar la expansión de Taylor de una función \(y : I \subset \mathbb{R} \to \mathbb{R}\) alrededor de un punto \(x_0 \in I\), donde \(y\) es solución a la ecuación diferencial de segundo orden

\begin{equation}
\label{eq:numerov-eq}
    y''(x) = - g(x)y(x) + s(x),
\end{equation}

con \(g: I \to \mathbb{R}\) y \(s: I \to \mathbb{R}\) como funciones conocidas, definidas en el mismo intervalo \(I\) que \(y\). Para que la expansión de Taylor sea válida hasta el orden requerido, se exige que \(y\) sea al menos \(C^4\) i.e. cuatro veces diferenciable. Además, \(g\) y \(s\) deben ser funciones suficientemente suaves e.g. continuas, para garantizar la existencia de una solución.

El método de Numerov sigue un esquema de diferencias finitas, por lo que comenzamos con una expansión de Taylor, a quinto órden. Para un paso adelante \((x= x_{0} + \Delta x)\) y un paso atrás \((x=x_{0} - \Delta x)\), se tiene

\[ y(x_{0} + \Delta x ) = y(x_{0}) + y'(x_{0})\Delta x + \frac{y''(x_{0})}{2!} \Delta x^2 + \frac{y'''(x_{0})}{3!}\Delta x^3 + \frac{y^{(4)}(x_{0})}{4!}\Delta x^4 + \frac{y^{(5)}(x_0)}{5!} + \order{\Delta x^6}, \]

\[ y(x_{0} - \Delta x ) = y(x_{0}) - y'(x_{0})\Delta x + \frac{y''(x_{0})}{2!} \Delta x^2 - \frac{y'''(x_{0})}{3!}\Delta x^3 + \frac{y^{(4)}(x_{0})}{4!}\Delta x^4 - \frac{y^{(5)}(x_0)}{5!} + \order{\Delta x^6}. \]

Se define una malla uniforme \(x_n = x_0 + n\Delta x\) y se denota:

\begin{itemize}
\item \(y(x_n) \equiv y_n\) (valores de la funcion en puntos de la malla).
\item \(y(x_n \pm \Delta x) \equiv y_{n \pm 1}\) (puntos adyacentes).
\end{itemize}

Así, las expansiones se reescriben como:

\[ y_{n+1} = y_n + y_n' \Delta x + \frac{1}{2} y_n'' (\Delta x)^2 + \frac{1}{6} y_n''' (\Delta x)^3 + \frac{1}{24} y_n^{(4)} (\Delta x)^4 + \frac{1}{120} y_n^{(5)} (\Delta x)^5 + \order{\Delta x^6} \]

\[ y_{n-1} = y_n - y_n' \Delta x + \frac{1}{2} y_n'' (\Delta x)^2 - \frac{1}{6} y_n''' (\Delta x)^3 + \frac{1}{24} y_n^{(4)} (\Delta x)^4 - \frac{1}{120} y_n^{(5)} (\Delta x)^5 + \order{\Delta x^6} \]

Al sumar ambos desarrollos se obtiene

\[ y_{n+1} + y_{n-1} = 2y_n + y_n''(\Delta x)^2 + \frac{1}{12} y_n^{(4)} (\Delta x)^4 + \mathcal{O} [(\Delta x)^6] \]

Luego, se puede escribir

\[ z_n \equiv y'' = - g_ny_n + s_n, \]

y aplicamos la expresion obtenida

\[ z_{n+1} + z_{n-1} = 2z_n + z_n'' (\Delta x)^2 + \mathcal{O}[(\Delta x)^4] \]

Esta es una expresion para la segunda derivada desarrollando en Taylor hasta tercer orden, de modo que, ademas, se obtiene una expresion para la cuarta derivada \(y^{(4)}\)

\[ z''_n = \frac{z_{n+1}+z_{n-1}-2z_n}{(\Delta x)^2} + \mathcal{O}[(\Delta x)^2] = y_n^{(4)}. \]

sustituyendo en la expresion

\[ y_{n+1} + y_{n-1} = 2y_n + y_n''(\Delta x)^2 + \frac{1}{12} y_n^{(4)} (\Delta x)^4 + \mathcal{O} [(\Delta x)^6] \]

se obtiene la formula de Numerov

 \begin{align*}
  &y_{n+1} = 2y_n - y_{n-1} + (z_n)(\Delta x)^2 + \frac{1}{12}(z_{n+1}+z_{n-1}-2z_n)(\Delta x)^2 + \order{(\Delta x)^6} \\
  &y_{n+1} =  2y_n -y_{n-1} + (-g_ny_n + s_n) (\Delta x)^2 + \frac{1}{12} [(-g_{n+1}y_{n+1}+s_{n+1}) \\
  & \qq{}\qq{} +(-g_{n-1}y_{n-1}+s_{n-1})-2(-g_ny_n+s_n)](\Delta x)^2 + \order{(\Delta x)^6} \\
  &y_{n+1}\qty[1 + \frac{1}{12}g_{n+1}(\Delta x)^2] = 2y_n \qty[1+\qty(\frac{-g_n}{2}+\frac{g_n}{12})(\Delta x)^2] \\
  & \qq{}\qq{} - y_{n-1}\qty[1+\frac{1}{12}g_{n-1}(\Delta x)^2] + \frac{1}{12}(s_{n+1}+10s_n+s_{n-1})(\Delta x)^2 + \order{(\Delta x)^6} \\
  &y_{n+1}\qty[1 + \frac{1}{12}g_{n+1}(\Delta x)^2] = 2y_n \qty[1-\frac{5}{12}g_n(\Delta x)^2] \\
  & \qq{}\qq{} -y_{n-1}\qty[1+\frac{1}{12}g_{n-1}(\Delta x)^2] + \frac{1}{12}(s_{n+1}+10s_n+s_{n-1})(\Delta x)^2 + \order{(\Delta x)^6}.
\end{align*}

Esta expresion nos permite propagar la solucion \(y_n\) conociendo los primeros dos valores \(y_0\) y \(y_1\). Las ecuaciones que se van a resolver por medio de esta formula tienen \(s(x)=0\). Luego, podemos simplificar la formula por medio de la cantidad

\[ f_n \equiv 1 + \frac{1}{12}g_n(\Delta x)^2, \]

haciendo

\begin{align*}
    f_n - 1 &= \frac{1}{12}g_n(\Delta x)^2 \\
    1-5(f_n -1) &= 1-\frac{5}{12}g_n(\Delta x)^2 \\
    6-5f_n &= 1-\frac{5}{12}g_n(\Delta x)^2 \\
    12-10f_n &= 2\qty(1-\frac{5}{12})g_n(\Delta x)^2
\end{align*}

se reescribe la formula de Numerov

\begin{equation}
\label{eq:numerov}
    \boxed{y_{n+1} = \frac{(12-10f_n)y_n-f_{n-1}y_{n-1}}{f_{n+1}}}
\end{equation}




Este metodo se aplica a los sistemas fisicos que se pueden modelar con una ecuacion diferencial del tipo \ref{eq:numerov-eq}. En particular vamos a resolver la ecuacion de Schrodinger unidimensional independiente del tiempo para el sistema de oscilador armonico y el atomo de hidrogeno, notamos que hay mas ecuaciones interesantes que se pueden resolver como la ecuacion de Poisson.
\section{Oscilador Harmonico}
\label{sec:orgfbc7872}

Al aplicar el metodo de separacion de variables la ecuacion de Schrodinger se obtiene la ecuacion de Schrodinger unidimensional para una funcion \(\psi = \psi(x)\)

\begin{equation}
\label{eq:schr1}
\frac{\partial^{2}\psi}{\partial x^2} = - \frac{2m}{\hbar^2}(E - V(x))\psi
\end{equation}

Un sistema de oscilador armonico es cuando el potencial en la ecuacion (\ref{eq:schr1}) es

\[ V(x) = - \frac{1}{2}K x^2. \]
donde \(K\) es una constante.

El desarrollo se simplifica en gran medida al hacer el cambio a unidades adimensionales:

\begin{itemize}
\item Variable adimensional \(\xi\)
\item Esta variable se relaciona con \(x\) por medio de la longitud \(\lambda\) de modo que \(x = \lambda \xi\)
\end{itemize}

\[ \pdv[2]{\psi}{(\lambda \xi)} = \qty( - \frac{2m E }{\hbar^2} + \frac{mK \lambda^2}{\hbar^2} \xi^2  )\psi \]

\[ \pdv[2]{\psi}{\xi} = \qty( - \frac{2m E \lambda^{2}}{\hbar^2} + \frac{mK \lambda^4}{\hbar^2} \xi^2  )\psi \]

\begin{itemize}
\item Hacemos \(mK\lambda^4 /\hbar^2 = 1\), de donde

\[ \lambda = (\hbar^2/mK)^{1/4} \]

\item Relacionamos la frecuencia angular del oscilador con la constante de fuerza

\[ \omega = \sqrt{\frac{K}{m}} \implies K = m\omega^2 \]

\item La variable adimensional queda

\[ \lambda\xi = x \implies \xi=\qty(\frac{mK}{\hbar^2})^{1/4} x = \qty(\frac{m \omega}{\hbar})^{1/2} x  \]

\item Introducimos la energia adimensional \(\epsilon\)

\[ \epsilon = \frac{2E}{\hbar \omega} \]

\item Sustituyendo estas expresiones en la ecuacion de Schrödinger

\[ \pdv[2]{\psi}{\xi} = \qty( - \frac{2m E \lambda^{2}}{\hbar^2} + \frac{mK \lambda^4}{\hbar^2} \xi^2  )\psi \]

\[ \pdv[2]{\psi}{\xi} = \qty( - \frac{2m (\epsilon \hbar \omega/2) (\hbar^2/m^2\omega^2)^{1/2}}{\hbar^2} +  \xi^2  )\psi \]

\item Finalmente la ecuacion de Schrödinger adimensional es:

\begin{equation}
\label{eq:ho}
\boxed{\pdv[2]{\psi}{\xi} = -2\qty(\epsilon - \frac{\xi^2}{2})\psi}
\end{equation}
\end{itemize}

con \(V(\xi) = \frac{1}{2}\xi^2\).
\subsection{Solucion Exacta}
\label{sec:orgf4bf1a0}

\subsubsection{Analisis asintotico}
\label{sec:org2788a5b}

Para grande \(\xi\), las soluciones de (\ref{eq:ho}), donde \(\epsilon\) se puede despreciar, son de la forma

\[ \psi(\xi) \sim \xi^n e^{\pm \xi^2/2}, \]

donde \(n\) cualquier valor finito. El exponente con signo positivo da lugar a funciones de onda no normalizables por lo que corresponde a soluciones no fisicas. Entonces asumimos que su comportamiento asintotico hace que la funcion de onda sea

\begin{equation}
\label{eq:ho_sol1}
\psi(\xi) = H(\xi)e^{-\xi^2/2}
\end{equation}

donde \(H(\xi)\) es alguna funcion bien comportada para \(\xi\) grande (de modo que el comportamiento asintotico este determinado por el factor \(e^{-\xi^2/2}\)). En particular \(H(\xi)\) no debe crecer como \(e^{\xi^2}\) para asi obtener soluciones fisicas. Bajo asumir que la funcion de onda es (\ref{eq:ho_sol1}), la ecuacion (\ref{eq:ho}) se convierte en una ecuacion para \(H(\xi)\):

\begin{gather*}
    \dv[2]{\xi}(H(\xi)e^{-\xi^2}/2) = -2\qty(\epsilon- \frac{\xi^2}{2})H(\xi)e^{-\xi^2/2} \\
    \dv[2]{H(\xi)}{\xi} e^{-\xi^2/2}  -\xi\dv{H(\xi)}{\xi}e^{-\xi^2/2} - \xi\dv{H(\xi)}{\xi}e^{-\xi^2/2} + \xi^2 H(\xi)e^{-\xi^2/2}- H(\xi) e^{-\xi^2/2} = -2\qty(\epsilon- \frac{\xi^2}{2})H(\xi)e^{-\xi^2/2} \\
    \dv[2]{H(\xi)}{\xi} - 2\xi \dv{H(\xi)}{\xi} + (2\epsilon - 1)H(\xi) = 0
\end{gather*}

Se expande la solucion \(H(\xi)\) en una serie de potencias

\[ H(\xi) = \sum_{n=0}^{\infty} A_n\xi^n \]

la primer derivada es simplemente

\[ \dv{H}{\xi} = \sum_{n=0}^{\infty} nA_n\xi^{n-1} \]

para la segunda derivada, diferenciamos cada termino

\[ \dv[2]{H}{\xi} = \dv{\xi}(A_1 + 2A_2\xi + 3A_3 \xi^2 + ... ) = 2A_2 + 2*3 A_3\xi + 3*4 A_4\xi^2 + ... = \sum_{n=0}^{\infty} (n+1)(n+2)A_{n+2} \xi^n \]
sustituyendo en la ecuacion para \(H(\xi)\) se tiene

\begin{align*}
    &\dv[2]{H(\xi)}{\xi} - 2\xi \dv{H(\xi)}{\xi} + (2\epsilon - 1)H(\xi) = 0 \\
    &\sum_{n=0}^{\infty} \{(n+1)(n+2)A_{n+2} \xi^n  - 2\xi(nA_n\xi^{n-1}) + (2\epsilon - 1)A_n\xi^n \}= 0 \\
    &\sum_{n=0}^{\infty}  \{(n+1)(n+1) A_{n+2} + (2\epsilon - 2n - 1)A_n \} \xi^n = 0
\end{align*}

esta expresion se debe satisfacer para todo \(\xi\) por el teorema de existencia y unicidad, entonces los coeficientes de todo orden deben ser cero:

\[ (n+2)(n+1)A_{n+2} + (2\epsilon - 2n -1)A_n  =0 \]

asi, dados \(A_0\) y \(A_1\), se puede determinar por recursion \(H(\xi)\) como una serie de potencias

\begin{equation}
\label{eq:rec-hermite}
     A_{n+2} = \frac{(2\epsilon - 2n -1)A_n}{n^2 + 3n + 2}
\end{equation}

\begin{gather*}
    \text{Para \(n\) muy grande, se tiene:} \\
    A_{n+2}  \sim \frac{2A_n}{n}
\end{gather*}

Se resuelve esta recursion para el caso par e impar:

\begin{itemize}
\item Para una potencia par \(n=2k\):

\[ A_{2k+2} \sim \frac{1}{k}A_{2k} \]

\item Iterando:

\[ A_{2k} \sim \frac{1}{k-1}A_{2k-2} \sim \frac{1}{k-1}\cdot \frac{1}{k-2}A_{2k-4}\sim \frac{A_0}{(k-1)!} \]

\item Usando \((k-1)! = k!/k\), para \(k\) muy grande, la solucion a la recursion es

\[ A_{2k} \sim \frac{A_0}{k!} \]

\item Similarmente, para una potencia impar \(n = 2k+1\):

\begin{gather*}
A_{2k+3} \sim \frac{2}{2k+1}A_{2k+1} \sim \frac{1}{k}A_{2k+1} \\
A_{2k+1} \sim \frac{A_1}{k!}
\end{gather*}

\item Por lo tanto, para \(n\) muy grande, la recursion se comporta como:
\end{itemize}

\[ A_n \sim \frac{1}{(n/2)!} \]

Esto implica:

\[ H(\xi) \sim \sum_k \qty[\frac{A_0}{k!}\xi^{2k} + \frac{A_1}{k!}\xi^{2k+1}] = A_0e^{\xi^2} + A_1\xi e^{\xi^2} \]

Esta expresion se interpreta como que la recurrencia (\ref{eq:rec-hermite}) produce una funcion \(H(\xi)\) que crece como \(e^{\xi^2}\) y da soluciones divergentes, i.e. no fisicas. Para prevenir este comportamiento, debemos truncar la serie despues de algun \(n\) y asi reducir la solucion a un polinomio de grado finito. Entonces, en la recursion (\ref{eq:rec-hermite}), para que la serie termine,

\begin{gather*}
      A_{n+2} = \frac{(2\epsilon - 2n -1)A_n}{(n+2)(n+1)} \\
      2\epsilon - 2n -1 = 0 \\
      \epsilon = n + \frac{1}{2}
\end{gather*}

donde \(n\) es un entero positivo. Esta condicion nos da la cuantizacion de la energia del oscilador harmonico:

\begin{equation}
\label{eq:ho-energy}
    E_n = (n+\frac{1}{2})\hbar \omega \qc n \in \mathbb{Z}^+
\end{equation}

Los polinomios correspondientes \(H_n(\xi)\) son los polinomios de Hermite, donde \(H_n(\xi)\):

\begin{itemize}
\item Es de grado \(n\) en \(\xi\)

\item Tiene \(n\) nodos

\item Es par para \(n\) par e impar para \(n\) impar
\end{itemize}

Finalmente, la funcion de onda correspondiente a la energia \(E_n\) es

\[ \psi_n(\xi) = H_n(\xi)e^{-\xi^2/2} \]
\section{Notacion y convenciones numericas}
\label{sec:org1db1dbd}

\subsection{Discretizacion y funciones continuas}
\label{sec:org2c1a08d}

Para representar funciones continuas \(\psi(x) : x\in I \subset \mathbb{R} \to \mathbb{R}\) de forma numerica:

\begin{itemize}
\item Se discretiza el dominio en una malla equiespaciada \(x_i = i*\Delta x\), donde

\[ i \in \{0,1,...,mesh\} \]

\item Los valores de \(\psi(x_i)\) se almacenan en un arreglo \(\psi_i\) indexado, donde

\[ \psi_i \equiv \psi(x_i) \to \psi[i] \text{ en Python} \]
\end{itemize}
\subsection{Implementacion en Python}
\label{sec:org0ac15c8}

Para una malla con \(mesh\) intervalos (\(mesh + 1\) puntos), el ultimo elemento del arreglo es

\begin{lstlisting}[language=Python,numbers=none]

mesh = 100
x = np.linspace(mesh+1)
psi = f(x) # Alguna funcion de x
psi[0] # Primer elemento de psi
psi[mesh] # Ultimo elemento de psi

\end{lstlisting}

\begin{itemize}
\item Como Python es cero indexado, hacer el tamano de la malla \(mesh + 1\) asegura que \(x_{max} = mesh * \Delta x\) y que \(\psi_{xmax} = psi[mesh]\)
\end{itemize}
\subsection{Notacion}
\label{sec:org2b2993a}

Usamos tres diferentes maneras de representar el mismo numero:

\[ \underbrace{\psi_R(x)}_{\text{Funcion que depende de \( x \)}} \to \underbrace{\psi_i^R}_{\text{iesimo valor de \( \psi^R \)}} \to \underbrace{psi\_R[i]}_{\text{Elemento \( i \) del arreglo \( psi\_R \)}} \]
\section{Algoritmo de biseccion para el oscilador harmonico}
\label{sec:org1a96e36}

\subsection{Descripcion general}
\label{sec:org42df0e8}

\begin{itemize}
\item \textbf{Inicializacion de la malla}

\begin{itemize}
\item Se discretiza el dominio espacial en \(mesh + 1\) puntos uniformemente espaciados \(x \in [0,xmax]\), donde \(xmax\) es un numero suficientemente grande para que la solucion \(\psi\) cumpla con las condiciones de frontera.

\item Se puede construir la funcion de onda para \(x\) negativo usando simetria, dado que \(\psi_n(-x) = (-1)^n\psi_n(x)\), lo cual es facilitado por la simetria del potencial de oscilador harmonico; de otro modo la integracion se tendria que dar sobre todo el intervalo \([-xmax, xmax]\)

\item El potencial de oscilador harmonico es

\[ V(x) = \frac{1}{2}x^2 \to V_i = 0.5*x^2_i \]
\item \hyperref[sec:orgdb4bc66]{Implementacion en Python}
\end{itemize}

\item \textbf{Busqueda de eigenvalores por biseccion}

\begin{itemize}
\item Cotas iniciales: \(e\_lower = min(V(x))\), \(e\_upper = max(V(x))\).
Con el objetivo de encontrar \(E\) tal que la solucion dada por la formula de Numerov \(\psi(x)\) sea fisica, e.g. suave, normalizable y cumple condiciones de frontera.
\end{itemize}

\item \textbf{Calculo del punto de retorno clasico}

\begin{itemize}
\item Se determina el primer indice donde \(V(x) > E\) usando la funcion auxiliar

\[ f^{aux} = 2(V-E) \frac{\Delta x^2}{12} \]

\item Detalles fisicos: \hyperref[sec:org7b13b2c]{Punto de retorno clasico}
\end{itemize}

\item \textbf{Integracion numerica de \(\psi(x)\)}

Se hacen dos integraciones

\begin{itemize}
\item Hacia afuera (\(0 \to icl\)): Se inicia \(\psi_0\) y \(\psi_1\) segun paridad y se propaga hasta \(\psi_{icl}\) con Numerov, asi propagando la solucion izquierda \(\psi^L\)

\item Hacia adentro (\(xmax \to icl\)): Se impone la condicion de frontera \(\psi(xmax) = 0\), luego se puede calcular \(\psi_{mesh -1}\) con la formula de Numerov y considerando que \(\psi_{mesh+1} = 0\); asi propagando la solucion derecha \(\psi^R\)

\item En general estas dos funciones tienen valores diferentes en \(x_c = icl * \Delta x\)

\item \hyperref[sec:org4d4f12b]{Ver integracion}
\end{itemize}

\item \textbf{Acoplamiento y normalizacion}

Queremos que la solucion dada por Numerov sea fisicamente valida, e.g. continua y normalizada:

\begin{itemize}
\item En \(x_c = icl * \Delta x\)
\begin{itemize}
\item Las soluciones \(\psi^L\) y \(\psi^R\) generalmente no coinciden en amplitud

\item Se escala \(\psi^R\) (asumiento que \(\psi^L\) es la solucion correcta) para garantizar continuidad en este punto de la malla

\[ \psi^{R} \leftarrow \psi^{R} \cdot \frac{\psi^{L}_{icl}}{\psi^{R}_{icl}} \]
\end{itemize}

\item Se calcula la norma, \(\mathcal{N}\), de \(\psi\) numericamente, i.e. regla del trapecio, tomando en cuenta la simetria

\[ \mathcal{N} = \int 2|\psi|^2 \dd{x} \]
y se normaliza

\[ \psi \to \frac{\psi}{\int 2|\psi|^2 \dd{x}} \]

\item Detalles de la implementacion \ref{sec:orgebecf68}
\end{itemize}

\item \textbf{Criterio de convergencia}

\begin{itemize}
\item Se calcula la discontinuidad de la derivada \(\Delta \psi'\) en \(icl\):

\item Actualizacion de energia:
\begin{itemize}
\item Si \(\Delta \psi' * \psi_{icl} > 0 \implies E\) es demasiado alto, entonces se actualiza la cota superior \(e\_upper = E\)
\item Si \(\Delta \psi' * \psi_{icl} < 0 \implies E\) demasiado bajo, entonces se actualiza la cota inferior \(e\_lower = E\)
\end{itemize}

\item \hyperref[sec:org2c822a5]{Detalles de convergencia}
\end{itemize}
\end{itemize}
\subsection{Malla}
\label{sec:orgdb4bc66}

\begin{itemize}
\item Utilizamos la libreria numpy para tener acceso a operaciones vectorizadas sobre los arreglos
\end{itemize}

\begin{lstlisting}[language=Python,numbers=none]

import numpy as np

x, dx = np.linspace(0, xmax, mesh+1, retstep=True)
vpot = 0.5 * x**2  # Potencial de oscilador harmonico

\end{lstlisting}
\subsection{Punto de retorno clasico}
\label{sec:org7b13b2c}

Es el punto, \(x_{rc}\), que marca el limite entre las regiones clasicamente permitida y prohibida:

\begin{itemize}
\item En \(x < x_{rc}\) ( \(V(x) < E\) ), \(\psi(x)\) oscila; los nodos se encuentran en esta region

\item En \(x> x_{rc}\) (\(V(x) > E\)), \(\psi(x)\) decae exponencialmente

\item Analizamos el comportamiento de \(f^{aux}\)

\[ f^{\text{aux}} = \frac{2(V-E) * \Delta x^2}{12} \]

\begin{itemize}
\item \(f^{\text{aux}} < 0\) en la region clasicamente permitida \(V(x)<E\)

\item \(f^{\text{aux}} > 0\) en la region prohibida \(V(x) > E\)

\item El cruce \(f^{\text{aux}} = 0\) coincide con \(V(x_{rc}) = E\) (punto de retorno clasico exacto), pero como \(f^{aux}\) vive en el espacio discretizado, no esta garantizado que \(f^{aux} = 0\) se cumpla, en otras palabras, no se garantiza que exista el indice exacto \(irc\) de modo que un punto en la malla \(x_{rc} = irc*\Delta x\) haga que \(f_{irc}^{aux} = 0\)
\end{itemize}
\end{itemize}
\subsubsection{ICL}
\label{sec:org45a925a}

\begin{itemize}
\item El indice \(icl\) es una aproximacion discreta al punto de retorno clasico \(x_{rc}\)

\begin{itemize}
\item Corresponde al primer punto de la malla \(x_c = icl * \Delta x\) donde \(V > E\)

\item El punto de retorno clasico exacto \(x_{rc}\) esta entre \(x_c - \Delta x\) y \(x_c\):

\[ x_{rc} \in [x_c-\Delta x, x_c] \]
\end{itemize}
\end{itemize}
\subsubsection{Implementacion}
\label{sec:org2d5772e}

\begin{lstlisting}[language=Python,numbers=none]

# Funcion auxiliar
f_aux = 2*(V-E) * (dx**2/12)

# Aseguramos que haya cambio de signo evitando el cero
f_aux = np.where(f == 0.0, 1e-20, f_aux)


# Deteccion de cambios de signo en f_aux
sign_changes = np.where(np.diff(np.sign(f_aux)))[0] # Devuelve los cambios de signo
icl = sign_changes[-1] + 1 # Primer punto en la region prohibida

\end{lstlisting}
\subsubsection{Notas}
\label{sec:org9bcdc05}

\begin{itemize}
\item La funcion auxiliar \(f^{aux}\) nos proporciona una relacion simple para determinar \(icl\). \textbf{La formula de Numerov usa:}

\[ f = 1 - f^{aux} \]

\item En el raro caso en el que \(f^{aux}\) sea exactamente igual a cero, no se detecta correctamente el cambio de signo. Utilizamos la funcion \(np.where(condition, x, y)\) para asegurar que se puedan detectar los cambios de signo correctamente:

\begin{itemize}
\item Es una funcion vectorizada que actua como un \(if-else\) sobre arreglos de NumPy

\item Para cada elemento en \(f^{aux}\)

\[\begin{cases}\text{Si } f^{aux}_i = 0.0, & f^{aux}_i \to 1e-20 \\ \text{Si } f^{aux}_{i} \neq 0.0, & f^{aux}_i \to f^{aux}_{i}\end{cases}\]
\end{itemize}

\item Cuando se usa sin los argumentos \((x,y)\), la funcion \(np.where(condition)\) devuelve una tupla de arreglos con los indices donde \(condition\) es \(True\). Para arreglos de \(1D\), el primer elemento de la tupla es el arreglo de estos indices, por ejemplo

\begin{lstlisting}[language=Python,numbers=none]

zeros_index = np.where(f == 0.0)[0] # Arreglo que contiene los indices donde f = 0

\end{lstlisting}

\item El objetivo es encontrar el primer punto donde \(f^{aux}\) cambia de negativo a positivo, i.e. la transicion de la region permitida a la prohibida. Se implementa en tres pasos:

\begin{itemize}
\item Calculo de signos

\begin{lstlisting}[language=Python,numbers=none]

signs = np.sign(f_aux)

\end{lstlisting}

Devuelve un arreglo con los signos de \(f^{aux}\)

\[\begin{cases} signs[i] = -1 & \text{si } f_i^{aux} < 0 \\ signs[i] = +1 & \text{si } f_i^{aux} > 0 \\ signs[i] = 0 & \text{si } f_i^{aux} = 0 \end{cases} \]

\item Luego:

\begin{lstlisting}[language=Python,numbers=none]

diffs = np.diffs(signs)

\end{lstlisting}

Es un arreglo con las diferencias entre elementos adyacentes de \(signs\). Es cero cuando no hay cambio de signo entre \(f^{aux}_i\) y \(f^{aux}_{i+1}\), cuando cambia de negativo a positivo, \(diffs[i] = +2\)

\item Finalmente, usamos el arreglo \(diffs\) como condiciones

\begin{lstlisting}[language=Python,numbers=none]

sign_changes = np.where(diffs)[0]

\end{lstlisting}

cuando \(diffs[i]\) es positivo, es como pasar \(True\) y guarda el indice \(i\) en el primer arreglo de la tupla que devuelve \(np.where()\). Accedemos al arreglo con el primer elemento de la tupla.
\end{itemize}

\item Asi, el primer elemento de \(f^{aux}\) en la region prohibida es

\begin{lstlisting}[language=Python,numbers=none]

icl = sign_changes[-1] + 1
f_aux[icl]

\end{lstlisting}
\end{itemize}
\subsection{Integracion}
\label{sec:org4d4f12b}

\subsubsection{Condiciones iniciales hacia afuera}
\label{sec:orgc142546}

La paridad de la funcion de onda determina los puntos iniciales para la recursion hacia adentro, se tiene que:

\[ \psi_n(-x) = (-1)^n\psi_n(x) \]

\begin{itemize}
\item Para \(n\) impar, los primeros dos puntos se inician como:
\begin{itemize}
\item \(\psi_0 = 0\)
\item \(\psi_1 = \dd{x}\) es un numero apropiadamente pequeno para las dimensiones del problema
\end{itemize}

\item Para \(n\) par:
\begin{itemize}
\item \(\psi_0 = 1\) es un numero arbitrario positivo apropiado para las dimensiones del problema. La magnitud es arbitraria ya que procedemos a normalizar la solucion

\item \(\psi_1\) se determina por la formula de Numerov \ref{eq:numerov}

\[ \psi_1 = \frac{(12 - 10 f_0\psi_0)-f_{-1}\psi_{-1}}{f_1} \]
donde \(f_{-1}\) y \(\psi_{-1}\) son el valor de \(f\) y \(\psi\) en \(x_{-1} = - \Delta x\), pero por simetria, se tiene que \((f_1,\psi_1) = (f_{-1},\psi_{-1})\) para obtener

\begin{align*}
    &\psi_1 = \frac{(12 - 10f_0\psi_0) - f_1y_1}{f_1} \\
    &f_1\psi_1 + f_1\psi_1 = (12 - 10f_0\psi_0) \\
    &\psi_1 = \frac{(12 - 10f_0\psi_0)}{2f_1}
\end{align*}
\end{itemize}
\end{itemize}
\subsubsection{Nodos y validacion de energia}
\label{sec:org48cef7a}

\begin{itemize}
\item La solucion \(\psi^L\) contiene todos los nodos de \(\psi(x)\), ya que \(\psi^R\) decae exponencialmente sin oscilar.

\begin{itemize}
\item Antes de acoplar \(\psi^L\) y \(\psi^R\), se verifica si \(\psi^L\) tiene el numero correcto de nodos para el \(n-\)esimo eigenvalor. Si no coincide, se ajustan las cotas de energia \((E_{min}, E_{max})\) y se reinicia el ciclo de biseccion.
\end{itemize}

\item La funcion \(outward\_ outward\) integration devuelve el conteo de cruces por cero \((ncross)\) para la validacion

\item Para este paso consideramos que \(\psi\) es la solucion en \([0,xmax]\), entonces el numero correcto de nodos se obtiene por simetria y considerando la paridad:

\begin{lstlisting}[language=Python,numbers=none]

# Adjuste de nodos por simetria basado en paridad del estado energetico
if nodes % 2 == 0:
    # Par
    ncross *= 2
else:
    # Impar
    ncross = 2 * ncross + 1

# Si los nodos no son correctos
# Actualizar cotas de energia
if ncross != nodes:
    if ncross > nodes:
        e_upper = e
    else:
        e_lower = e
    continue  # Reiniciar ciclo de biseccion

\end{lstlisting}
\end{itemize}
\subsubsection{Condiciones iniciales hacia adentro}
\label{sec:orgdede203}

Invertimos el orden de integracion, empezamos en la frontera hasta \(icl\). La formula de Numerov funciona igual que la integracion hacia afuera, entonces necesitamos los ultimos dos puntos \(\psi_{malla}\) y \(\psi_{malla-1}\):

\begin{itemize}
\item Para \(\psi_{malla}\), si aplicamos directamente la condicion de frontera \(\psi(xmax) = 0\), la recursion da un arreglo de puros ceros, entonces iniciamos con un valor arbitrariamente pequeno, apropiado para las dimensiones del problema, que en este caso es el paso de la malla

\[ \psi_{malla} = \dd{x} \]

\item Para \(\psi_{malla-1}\) usamos directamente la formula de Numerov, con \(\psi_{malla + 1} = 0\)

\[ \psi_{malla-1} = \frac{(12-10f_{malla})\psi_{malla}}{f_{malla-1}} \]
\end{itemize}

Usando estos valores podemos propagar el resto de \(\psi^R\) hasta el punto de retorno clasico.
\subsubsection{Implementacion}
\label{sec:orgc65af15}

\begin{itemize}
\item Integracion hacia afuera:

\begin{itemize}
\item Durante la integracion hacia adentro, el valor de \(\psi_{icl}\) (calculado previamente en la integracion hacia afuera) se sobreescribe. En general estos valores son diferentes y crean una discontinuidad en la solucion. Guardamos \(\psi^L_{icl}\) en una variable auxiliar \(psi\_icl\) antes de iniciar la integracion hacia adentro

\begin{lstlisting}[language=Python,numbers=none]
psi = np.zeros_like(x)
# Iniciacion de psi basado en paridad
if nodes % 2:
    # Impar
    psi[0] = 0.0
    psi[1] = dx
else:
    # Par
    psi[0] = 1.0
    psi[1] = (6.0 - 5.0 * f[0]) * psi[0] / f[1]

psi_icl, ncross = outward_integration(psi, f, icl)

def outward_integration(psi, f, icl):
    ncross = 0
    for i in range(1, icl):
        psi[i+1] = ((12.0 - 10.0)*f[i] * psi[i] - f[i-1] * psi[i-1]) / f[i+1]
        ncross += (psi[i] * psi[i+1] < 0.0)  # Boolean to int
    return psi[icl], ncross

\end{lstlisting}
\end{itemize}

\item Integracion hacia adentro:

\begin{itemize}
\item La solución general cerca de \(xmax\) es una superposicion de exponenciales crecientes/decrecientes. El rescale suprime artificialmente la parte divergente, i.e. no física, preservando el decaimiento exponencial válido en la region clasicamente prohibida

\begin{lstlisting}[language=Python,numbers=none]

# Inward integration on the tail: initialize boundary conditions.
psi[-1] = dx
psi[-2] = f_10[-1] * psi[-1] / f[-2]

inward_integration(psi, f, icl, mesh)

def inward_integration(psi, f, icl, mesh):
    # Inward integration in [xmax, icl]
    for i in range(mesh-1, icl, -1):
        psi[i-1] = ((12.0 - 10.0)*f[i] * psi[i] - f[i+1] * psi[i+1]) / f[i-1]
        if abs(psi[i-1]) > 1e10:
            psi[i-1:-2] /= psi[i-1] # Rescale para suprimir comportamiento divergente

\end{lstlisting}
\end{itemize}
\end{itemize}
\subsection{Acoplamiento y normalizacion}
\label{sec:orgebecf68}

\subsubsection{Implementacion}
\label{sec:orgb8de890}

\begin{itemize}
\item Es una simple funcion que empalma \(\psi^R\) con \(\psi^L\) en \(x_c = icl*\Delta x\) escalando \(\psi^R\) por el factor

\[ \frac{\psi_{icl}^L}{\psi_{icl}^R} \]

\item Al tener una solucion continua \(\psi\) procedemos a normalizar:
\begin{itemize}
\item Como estamos aprovechando la simetria para solo calcular una mitad de \(\psi\), al normalizar tenemos que considerar que

\[ \int |\psi|^2 \dd{x} = \frac{1}{2} \int |\psi_{full}|^2 \dd{x} \]
\end{itemize}
\end{itemize}

\begin{lstlisting}[language=Python,numbers=none]

# Normalizar funcion de onda
scale_normalize_ho(psi, psi_icl, icl, x)

def scale_normalize_ho(psi, psi_icl, icl, x):
    # Match wavefunction at icl and normalize
    scaling_factor = psi_icl / psi[icl]
    psi[icl:-2] *= scaling_factor

    norm = np.sqrt(np.trapezoid(2*psi**2, x))  # Symmetric normalization
    psi /= norm

\end{lstlisting}
\subsection{Criterio de convergencia}
\label{sec:org2c822a5}

\subsubsection{Determinacion de discontinuidad en la primer derivada}
\label{sec:org2ba55d9}

Al haber normalizado la funcion de onda, nuestra solucion, en general tendra una discontinuidad en su primera derivada, que podemos expresar como

\[ {\psi'}_{icl}^{R} - {\psi'}_{icl}^{L} \]

Esta diferencia debe ser cero para una solucion apropiada, lo que solo ocurre cuando \(E\) esta muy cerca de ser un eigenvalor \(E_n\). El signo de la diferencia nos ayuda a entender si la energia de prueba \(E\) es muy alta o muy baja, para asi hacer una actualizacion apropiada en el metodo de biseccion.

Con \(i = icl\), calculamos la discontinuidad en la primera derivada usando las expansiones de Taylor:

\[ \psi_{i-1}^L = \psi_i^L - {\psi'}_i^L \Delta x + \frac{1}{2}{\psi''}_i^L (\Delta x)^2 + \order{(\Delta x)^3} \]
\[ \psi_{i+1}^L = \psi_i^L + {\psi'}_i^L \Delta x + \frac{1}{2}{\psi''}_i^L (\Delta x)^2 + \order{(\Delta x)^3} \]

Sumando estas dos expresiones, tomando en cuenta que al haber acoplado las recursiones, se tiene \(\psi_i^L = \psi_i^R = \psi_i\), y que \({\psi''}_i^L = {\psi''}_i^{R} = -g_i\psi_i\), por el metodo de Numerov:

\[ \psi^L_{i-1} + \psi^R_{i+1} = 2\psi_i + ({\psi'}_i^R - {\psi'}_i^L)\Delta x - g_i\psi_i(\Delta x)^2 + \order{(\Delta x)^3} \]

esto es

\[ {\psi'}_i^R - {\psi'}_i^L = \frac{\psi_{i-1}^L + \psi_{i+1}^R - [2-g_i(\Delta x)^2]\psi_i}{\Delta x} + \order{(\Delta x)^2} \]

en terminos de \(f\), obtenemos la expresion para la discontinuidad en la primer derivada:

\[ {\psi'}_i^R - {\psi'}_i^L = \frac{\psi_{i-1}^L + \psi_{i+1}^R - [14-12f_i]\psi_i}{\Delta x} + \order{(\Delta x)^2} \]
\subsubsection{Logica de biseccion}
\label{sec:org986608a}

La discontinuidad en la derivada

\[ {\psi'}_i^R - {\psi'}_i^L \equiv ddelta  \]

indica como la solucion \(\psi\) se desvia del decaimiento exponencial fisico en \(x>x_{rc}\). Su signo determina si \(E\) es demasiado alta o baja.

\begin{itemize}
\item Para que \(\psi\) sea una solucion fisica:

\begin{itemize}
\item En la region clasicamente prohibida \(x> x_{rc}\), la solucion correcta debe decaer exponencialmente

\item Para que \(\psi\) decaiga exponencialmente en la region clasicamente prohibida, la magnitud de \({\psi'}^R\) debe ser mayor que la de \({\psi'}^L\), lo que implica que la pendiente de la recta tangente en esta region es mas pronunciada.
\end{itemize}

\item El signo de \(\psi_i\) esta determinado por la paridad del estado energetico, con paridad par, \(\psi_i > 0\) y con paridad impar \(\psi_i< 0\). Luego el signo de \(ddelta\) nos dice:
\end{itemize}

\begin{center}
\begin{tabular}{lll}
Paridad \(\psi_i\) & \(ddelta > 0\) & \(ddelta < 0\)\\
\hline
Par & Decae muy lento (\(E\) muy alta) & Decae muy rapido (\(E\) muy baja)\\
Impar & Decae muy rapido (\(E\) muy baja) & Decae muy lento (\(E\) muy alta)\\
\hline
\end{tabular}
\end{center}

\begin{itemize}
\item El producto \(ddelta * \psi_i\) codifica estos cuatro casos correctamente:

\begin{itemize}
\item Si \(ddelta * \psi_i > 0\), ajustamos las cotas a la mitad inferior del intervalo

\item Si \(ddelta * \psi_i < 0\) ajustamos las cotas a la mitad superior del intervalo
\end{itemize}

\item Se declara convergencia cuando:

\[ e\_upper - e\_lower < tol \]

donde \(tol\) es una tolerancia arbitraria. La energia final es

\[ E_n \approx 0.5 * (e\_upper + e\_lower) \]
\end{itemize}
\subsubsection{Implementacion}
\label{sec:org92fc76f}

\begin{lstlisting}[language=Python,numbers=none]

# Compute the derivative discontinuity at the matching point
ddelta = (psi[icl+1] + psi[icl-1] - (14.0 - 12.0 * f[icl]) * psi[icl]) / dx

# Check convergence: update energy bounds based on the sign of the discontinuity.
if (e_upper - e_lower) < tol:
    break

if ddelta * psi[icl] > 0.0:
    e_upper = e
else:
    e_lower = e

\end{lstlisting}
\section{Algoritmo de biseccion para el atomo de hidrogeno}
\label{sec:org4a1872f}

Para resolver la ecuacion radial modificamos ligeramente el algoritmo

\begin{itemize}
\item Al hacer la transformacion a coordenadas logaritmicas, ajustamos la malla

\begin{itemize}
\item \(x = \ln(Z*r)\), donde \(r\) es la coordenada radial fisica
\end{itemize}

\item El potencial efectivo es

\begin{lstlisting}[language=Python,numbers=none]
v_eff = -2*Z/r + l*(l + 1)/r**2
\end{lstlisting}

\item Iniciacion de la funcion de onda tomando en cuenta su comportamiento asintotico

\begin{lstlisting}[language=Python,numbers=none]

psi[0] = (r[0] ** (l + 1)) * (1 - (2 * Z * r[0])/(2 * l + 2)) / np.sqrt(r[0])
psi[1] = (r[1] ** (l + 1)) * (1 - (2 * Z * r[1])/(2 * l + 2)) / np.sqrt(r[1])

psi_icl, ncross = outward_integration(psi, f, f_10, icl)

\end{lstlisting}

\item Usando \(\psi_0\) y \(\psi_1\) propagamos la solucion con el paso constante de la malla logaritmica \(\dd{x}\) usando la formula de Numerov

\item Antes de hacer la integracion hacia adentro, consideramos que la solucion debe tener \(n - l - 1\) nodos

\begin{lstlisting}[language=Python,numbers=none]

# Checar que las cotas contienen el numero correcto de nodos
nodes_expected = n - l - 1
if ncross != nodes_expected:
    if ncross > nodes_expected:
        e_upper = e
    else:
        e_lower = e
    e = 0.5 * (e_lower + e_upper)
    continue  # Saltarse la integracion hacia adentro si la cantidad de nodos es incorrecta

\end{lstlisting}

\item Para la integracion hacia adentro, iniciamos los ultimos dos valores de la misma manera

\begin{lstlisting}[language=Python,numbers=none]

psi[-1] = dx
psi[-2] = f_10[-1] * psi[-1] / f[-2]
inward_integration(psi, f, icl, mesh, f_10)

\end{lstlisting}

\item La normalizacion se hace en coordenadas esfericas

\begin{lstlisting}[language=Python,numbers=none]

norm = np.sqrt(np.trapezoid(psi**2 * r**2, x))  
psi /= np.sqrt(norm)

\end{lstlisting}
\end{itemize}
\section{Teoria de perturbaciones}
\label{sec:org52ec7a4}

Para mejorar la convergencia, podemos utilizar teoria de perturbaciones para obtener correcciones de energia, asi cambiamos el criterio de convergencia para cuando la correccion en la energia se mas pequena que la tolerancia, usamos la funcion:

\begin{lstlisting}[language=Python,numbers=none]
def update_energy(icl, f, psi, dx, ddx12, e, e_lower, e_upper):
    # ddx12 es una constante auxiliar -> dx**2 / 12.0
    # f -> 1 + ddx12 * (lnhfsq + r2 * (e - vpot)) dada por Numerov
    # icl es el primer indice en la region prohibida
    i = icl
    psi_cusp = (psi[i-1] * f[i-1] + psi[i+1] * f[i+1] + 10 * f[i] * psi[i]) / 12.0
    dfcusp = f[i] * (psi[i] / psi_cusp - 1.0)
    de = dfcusp / ddx12 * (psi_cusp ** 2) * dx

    if de > 0:
        e_lower = e
    elif de < 0:
        e_upper = e
    e += de
    e = max(min(e, e_upper), e_lower)
    return e, e_lower, e_upper, de # Devuelve la energia correjida y nuevas cotas
\end{lstlisting}

\begin{itemize}
\item Usando la formula de Numerov, calculamos \(\psi_{icl}\) tomando \(icl-1\) o \(icl+1\) como punto central, luego tenemos la cantidad \(psicusp\) que es el valor calculado usando \(icl\) como punto central. En general \(psicusp \neq \psi_{icl}\)

\item Usando teoria de perturbaciones, consideramos a la funcion donde \(psicusp \neq \psi_{icl}\) como la solucion exacta a un problema diferente; uno en el que se superimpone una funcion delta \(v_0\delta(x-x_c)\) en \(x_c \equiv x_{icl}\) (porque estamos en la malla logaritmica) al potencial. La presencia de una delta causa una discontinuidad en la primera derivada y el tamano de la discontinuidad esta relacionada al coeficiente de la delta. Al conocer este coeficiente, podemos dar una estimacion, basada en teoria de perturbaciones, de la diferencia entre el eigenvalor actual (para este potencial diferente) y el eigenvalor para el potencial que nos interesa

\item Para lidiar con la delta en una integracion numerica, asumimos que solo tiene un valor en el intervalo \(\Delta x\) centrado en \(\psi_{icl}\).

\item La formula de Numerov que usamos es

\[ \psi_{i+1}f_{i+1} = (12-10f_i)\psi_i - f_{i-1}\psi_{i-1} \]   

donde normalmente extraemos \(\psi_{i+1}\) dividiento entre \(f_{i+1}\), pero ahora suponemos que \(f_{icl}\) tiene un valor diferente que no conocemosd \(fcusp\), de modo que nuestra funcion satisface la formula de Numerov tambien en un punto \(icl\). Entonces se debe cumplir lo siguiente:

\[ fcusp * psicusp = f_{icl}*\psi_{icl} \]

ya que este producto esta dado por el metodo de Numerov integrando desde \(icl-1\) hasta \(icl+1\) y \(psicusp\) es el valor que la funcion \(\psi\) debe tener para que satisfaga la formula de Numerov tambien en \(icl\). Como consecuencia, el valor de \(dfcusp\) es solamente \(fcusp-f_{icl}\), o bien \(\delta f\)

\item El siguente paso es calcular la variacion \(\delta V\) del potencial \(V(r)\) correspondiente a \(\delta f\). De la ecuaciond de Numerov \(\psi''(x) = g(x)\psi(x)\), para la ecuacion radial tenemos

\[ g(x) = \frac{2m}{\hbar^2}r^2(x) (E-V(r(x))) - (l+\frac{1}{2})^2 \]

diferenciando se obtiene

\[ \delta g(x) = - (2m/\hbar^2)r^2 \delta V \]

dado que \(f(x) = 1+g(x) (\Delta x^2/12)\), tenemos \(\delta g = (12/\Delta x^2) \delta f\), por lo que

\[ \delta V = - \frac{\hbar^2}{2m}\frac{1}{r^2} \frac{12}{\Delta x^2}\delta f \]

\item Teoria de perturbaciones a primer orden da la variacion del eigenvalor correspondiente

\[ \delta e = \ev{\delta V}{\phi} = \int |\psi(x)|^2r(x)^2 \delta V \dd{x} \]

donde \(\phi\) es la solucion radial completa y \(\psi\) es la funcion auxiliar que estamos trabajando

\item Escribimos la integral como una suma finita sobre puntos de la malla, con una sola contribucion diferente de cero proveniente de la region \(\Delta x\) centrada en el punto \(x_c = x_{icl}\). Finalmente la correccion de energia:

\[ \delta e = |\psi(x_c)|^2r(x_c)^2 \delta V \Delta x = - \frac{\hbar^2}{2m}\frac{12}{\Delta x^2}|\psi(x_c)|^2 \Delta x \delta f \]

esta es la expresion que utilizamos para calcular diferencia entre el eigenvalor para el potencial con la delta superimpuesta y el potencial correcto. Ya que en el primer paso, esta aproximacion puede tener errores muy grandes, usamos

\begin{lstlisting}[language=Python,numbers=none]
e = max(min(e, e_upper), e_lower)
\end{lstlisting}

para que la nueva energia no se salga de las cotas de energia.
\end{itemize}
\section{Central Potentials}
\label{sec:orgaa39a7c}

\[ H = \qty[ - \frac{\hbar^2}{2m}\nabla^2 + V(r)] \]
\subsection{Radial equaiton}
\label{sec:orgfdf4f36}

asdd

La probabilidad \(p(r) \dd{r}\) de encontrar a una particula a una distancia entre \(r\) y \(r + \dd{r}\) del centro esta dada por la integracion sobre solamente las variables angulares del cuadrado de la funcion de onda

\[ p(r) \dd{r} = \int_{\Omega} |\psi_{nlm}(r,\theta,\phi)|^2 r \dd{\theta}r\sin\theta\dd{\phi} \dd{r} = |R_{nl}|^2 r^2 \dd{r} = |\chi_{nl}|^2 \dd{r}  \]

donde introducimos la funcion auxiliar \(\chi(r)\), que se conoce como funcion de onda orbital

\[ \chi(r) = rR(r) \]

como consecuencia de la normalizacion de los harmonicos esfericos

\[ \int_0^{2\pi} \dd{\phi} \int_0^{\pi}\dd{\theta}|Y_{lm}(\theta,\phi|^2 \sin\theta = 1, \]

la condicion de normalizacion para \(\chi\) es

\[ \int_0^{\infty} | \chi_{nl}(r)|^2 \dd{r} = 1. \]

Esto significa que la funcion \(|\chi(r)|^2\) se puede interpretar directamente como la densidad de probabilidad radial. Entonces escribimos la ecuacion radial para \(\chi(r)\) en vez de para \(R(r)\)

\begin{equation}
\label{eq:schr-rad}
- \frac{\hbar^2}{2m}\dv[2]{\chi}{r} + \qty[V(r) + \frac{\hbar^2 l (l+1)}{2mr^2} - E] \chi(r) = 0
\end{equation}

esta es la forma de la ecuacion de Schrodinger unidimensional para una particula bajo un potencial efectivo

\[ \hat{V}(r) = V(r) + \frac{\hbar^2 l(l+1)}{2mr^2} \]
\subsection{Malla Logaritmica}
\label{sec:orgda17906}

\[ x = x(r) \]

La relacion entre la malla con paso constante \(\Delta x\) y la malla de paso variable esta dada por

\[ \Delta x  = x'(r) \Delta r \]

La malla logaritmica toma la forma

\[ x(r) \equiv \ln(\frac{Zr}{a_0}) \]

y se obtiene

\[ \Delta x = \frac{\Delta r}{r} \]

La razon \(\Delta r / r\) se mantiene constante en la malla de \(r\). Al transformar la ecuacion \ref{eq:schr-rad} en la nueva variable \(x\)

\begin{itemize}
\item Expresando las derivadas con respecto a \(r\) en terminos de derivadas con respecto a \(x\). Dado que:

\[ x = \ln(\frac{Zr}{a_0}) \implies r = \frac{a_0}{Z}e^x \]

\item La primera derivada de \(\chi\) con respecto a \(r\) es

\[ \dv{\chi}{r} = \dv{\chi}{x}\cdot \dv{x}{r} \]

luego

\begin{align*}
    \dv{x}{r} &= \dv{r}\ln(\frac{Zr}{a_0}) = \frac{1}{r} \\
    \dv{\chi}{r} &= \frac{1}{r} \dv{\chi}{x} \\
    \dv[2]{\chi}{r} &= \frac{1}{r^2} \dv{\chi}{x} + \frac{1}{r}\dv{r}(\dv{\chi}{x}) \\
    \dv{r}(\dv{\chi}{x}) &= \dv[2]{\chi}{x} \cdot \dv{x}{r} = \frac{1}{r} \dv[2]{\chi}{x} \\
    \dv[2]{\chi}{x} &= - \frac{1}{r^2}\dv{\chi}{x} + \frac{1}{r^2}\dv[2]{\chi}{x} = \frac{1}{r^2}\qty(\dv[2]{\chi}{x} - \dv{\chi}{x})
\end{align*}

\item Sustituyendo en la ecuacion \ref{eq:schr-rad}

\begin{align*}
&- \frac{\hbar^2}{2m} \frac{1}{r^2} \qty(\dv[2]{\chi}{x}-\dv{\chi}{x}) + \qty[V(r) + \frac{\hbar^2 l(l+1)}{2mr^2} - E r^2]\chi = 0 \\
&- \frac{\hbar^2}{2m}\qty(\dv[2]{\chi}{x}-\dv{\chi}{x}) + \qty[V(r) r^2 + \frac{\hbar^2 l(l+1)}{2m} - E r^2]\chi = 0 \\
&- \frac{\hbar^2}{2m}\dv[2]{\chi}{x} + \frac{\hbar^2}{2m}\dv{\chi}{x} + \qty[V(r)r^2 + \frac{\hbar^2 l(l+1)}{2m}-Er^2]\chi = 0
\end{align*}

\item Expresando \(r\) en terminos de \(x\)

\[ - \frac{\hbar^2}{2m}\dv[2]{\chi}{x} + \frac{\hbar^2}{2m}\dv{\chi}{x} + \qty[V\qty(\frac{a_0}{Z}e^x)\qty(\frac{a_0}{Z})^2 e^{2x} + \frac{\hbar^2l(l+1)}{2m}- E\qty(\frac{a_0}{Z})^2e^{2x}]\chi = 0 \]

\item En esta ecuacion aparece un termino de la primera derivada

\[ \frac{\hbar^2}{2m} \dv{\chi}{x} \]

en consecuencia, no se pueden utilizar los metodos de integracion convencionales para esta ecuacion.
\end{itemize}

Para obtener una ecuacion que se pueda integrar sobre una malla logaritmica, tenemos que hacer la transformacion

\[ y(x) = \frac{1}{\sqrt{r}}\chi(r(x)) \]

\[ \dv[2]{y}{x} + \qty[\frac{2m_e}{\hbar^2}r^2(E-V(r)) - \qty(l + \frac{1}{2})^2]y(x) = 0 \]
\end{document}
