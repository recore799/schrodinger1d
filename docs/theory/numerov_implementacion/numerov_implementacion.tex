\documentclass[11pt]{article}
\usepackage[utf8]{inputenc}
\usepackage[spanish]{babel}
\usepackage[T1]{fontenc}
\usepackage{geometry}
\usepackage{amsmath,amssymb,bm,physics}
\usepackage{listings}
\usepackage{hyperref}
\usepackage{geometry}
\geometry{a4paper,margin=1in}

\title{Documentación del Método de Numerov}
\date{}

\begin{document}
\maketitle

\begin{abstract}
Esta documentación describe detalladamente la implementación numérica del método de Numerov 
para resolver la ecuación de Schrödinger unidimensional, aplicada tanto al oscilador armónico 
como al átomo de hidrógeno. Se discuten las decisiones de diseño, la notación, las condiciones 
de frontera, el algoritmo de bisección, los criterios de convergencia y las correcciones por 
teoría de perturbaciones. Además, se analiza la transformación a una malla logarítmica para el 
caso radial.
\end{abstract}

\section{Notación y convenciones numéricas}
\label{sec:notation}

\subsection{Discretización y funciones continuas}
Para representar funciones continuas \(\psi(x) : x\in I \subset \mathbb{R} \to \mathbb{R}\) de forma numérica:

\begin{itemize}
\item Se discretiza el dominio en una malla equiespaciada \(x_i = i\,\Delta x\), donde
\[ i \in \{0,1,\ldots,mesh\}. \]
\item Los valores de \(\psi(x_i)\) se almacenan en un arreglo \(\psi_i\) indexado:
\[ \psi_i \equiv \psi(x_i) \to \psi[i] \text{ en Python}. \]
\end{itemize}

\subsection{Implementación en Python}
Para una malla con \(mesh\) intervalos (\(mesh + 1\) puntos), el último elemento del arreglo es:

\begin{lstlisting}[language=Python]
mesh = 100
x = np.linspace(mesh+1)
psi = f(x)
psi[0]      # Primer elemento
psi[mesh]   # Ultimo elemento
\end{lstlisting}

Como Python es cero indexado, definir la malla con \(mesh + 1\) puntos asegura que 
\(x_{\text{max}} = mesh \,\Delta x\) y que \(\psi_{\text{max}} = \psi[mesh]\).

\subsection{Notación}
Usamos tres formas equivalentes de representar el mismo número:
\[
\underbrace{\psi_R(x)}_{\text{Función de }x}
\to
\underbrace{\psi_i^R}_{\text{Valor i-ésimo}}
\to
\underbrace{psi\_R[i]}_{\text{Elemento del arreglo}}.
\]

\section{Algoritmo de bisección para el oscilador armónico}
\label{sec:ho-bisection}

\subsection{Descripción general}
El método se basa en discretizar el dominio espacial y ajustar la energía \(E\)
mediante bisección hasta encontrar una solución continua y normalizable de la ecuación de Schrödinger.

\begin{itemize}
\item \textbf{Inicialización de la malla:}  
  Se discretiza \(x\in[0,x_{\text{max}}]\) en \(mesh+1\) puntos equiespaciados.  
  Gracias a la simetría del potencial armónico \(\psi_n(-x)=(-1)^n\psi_n(x)\), 
  basta integrar en \([0,x_{\text{max}}]\).

  \[
  V(x)=\frac{1}{2}x^2 \;\to\; V_i=0.5\,x_i^2.
  \]

\item \textbf{Búsqueda de eigenvalores por bisección:}  
  Se definen cotas iniciales \(E_{\text{min}}=\min(V)\), \(E_{\text{max}}=\max(V)\), 
  buscando \(E\) tal que la solución \(\psi(x)\) sea suave, normalizable y cumpla las condiciones de frontera.

\item \textbf{Cálculo del punto de retorno clásico:}  
  Se identifica el primer índice donde \(V(x)>E\) usando la función auxiliar:
  \[
  f^{aux}=2(V-E)\frac{\Delta x^2}{12}.
  \]

\item \textbf{Integración numérica de \(\psi(x)\):}  
  Se realizan dos integraciones:
  \begin{itemize}
  \item \emph{Hacia afuera} \((0\to icl)\): se propaga con Numerov desde las condiciones iniciales.
  \item \emph{Hacia adentro} \((x_{\text{max}}\to icl)\): se impone \(\psi(x_{\text{max}})=0\)
  y se retropropaga.
  \end{itemize}
  Las soluciones \(\psi^L\) y \(\psi^R\) se acoplan en el punto \(x_c=icl\,\Delta x\).

\item \textbf{Acoplamiento y normalización:}  
  Se escalan las soluciones para garantizar continuidad en \(x_c\):
  \[
  \psi^R \leftarrow \psi^R \frac{\psi^L_{icl}}{\psi^R_{icl}},
  \]
  y se normaliza:
  \[
  \mathcal{N}=\int 2|\psi|^2\,dx, \qquad \psi\to \frac{\psi}{\mathcal{N}}.
  \]

\item \textbf{Criterio de convergencia:}  
  Se calcula la discontinuidad de la derivada \(\Delta\psi'\) en \(icl\).  
  Si \(\Delta\psi'\psi_{icl}>0\), la energía es demasiado alta;  
  si \(\Delta\psi'\psi_{icl}<0\), es demasiado baja.  
  Se actualizan las cotas hasta cumplir:
  \[
  E_{\text{max}}-E_{\text{min}}<tol.
  \]
\end{itemize}

\section{Punto de retorno clásico}
El punto \(x_{rc}\) marca el límite entre las regiones clásicamente permitida y prohibida:
\[
V(x_{rc})=E.
\]
En \(x<x_{rc}\), \(\psi\) oscila; en \(x>x_{rc}\), decae exponencialmente.
El índice \(icl\) aproxima este punto en la malla discreta.

\section{Integración y condiciones iniciales}
La paridad del estado define las condiciones iniciales:

\[
\psi_n(-x)=(-1)^n\psi_n(x).
\]
\begin{itemize}
\item Estados impares: \(\psi_0=0\), \(\psi_1=\Delta x\).
\item Estados pares: \(\psi_0=1\),
\(\psi_1=\frac{(12-10f_0\psi_0)}{2f_1}\).
\end{itemize}

\section{Criterio de convergencia}
Tras acoplar y normalizar, se evalúa la discontinuidad:
\[
\psi'^{R}_{icl}-\psi'^{L}_{icl}
=\frac{\psi_{icl+1}^R+\psi_{icl-1}^L-[14-12f_i]\psi_{icl}}{\Delta x}.
\]
El signo de esta cantidad determina si \(E\) debe aumentar o disminuir.

\section{Algoritmo de bisección para el átomo de hidrógeno}
Se usa una malla logarítmica:
\[
x=\ln(Zr), \qquad r=\frac{e^x}{Z}.
\]
El potencial efectivo:
\[
V_{\text{ef}}=-2\frac{Z}{r}+\frac{l(l+1)}{r^2}.
\]
El método procede de manera análoga al caso armónico, ajustando las condiciones iniciales según el comportamiento asintótico de la función radial.

\section{Corrección de energía por teoría de perturbaciones}
Para acelerar la convergencia se calcula una corrección de energía basada en la discontinuidad local del término \(f\) en el punto de acoplamiento.  
Usando teoría de perturbaciones de primer orden:
\[
\delta e = |\psi(x_c)|^2 r(x_c)^2 \delta V \Delta x 
= - \frac{\hbar^2}{2m}\frac{12}{\Delta x^2}|\psi(x_c)|^2 \Delta x\,\delta f.
\]
Esto permite estimar la diferencia entre el eigenvalor actual y el exacto para el potencial sin perturbación.

\section{Transformación a malla logarítmica}
Para la ecuación radial:
\[
x(r)=\ln\frac{Zr}{a_0}, \qquad \Delta x=\frac{\Delta r}{r}.
\]
Al reescribir la ecuación de Schrödinger en esta variable, aparece un término de derivada primera, por lo que se redefine:
\[
y(x)=\frac{1}{\sqrt{r}}\chi(r(x)),
\]
obteniendo una forma integrable:
\[
\dv[2]{y}{x}+\qty[\frac{2m_e}{\hbar^2}r^2(E-V(r))-(l+\tfrac{1}{2})^2]y(x)=0.
\]

\section*{Conclusión}
El método de Numerov, aplicado al oscilador armónico y al átomo de hidrógeno,
constituye una técnica precisa y estable para resolver la ecuación de Schrödinger unidimensional.  
Cada detalle de implementación —desde la elección de malla hasta el criterio de convergencia—
refleja directamente los principios físicos del sistema.  
Esta documentación busca servir como referencia técnica y como guía pedagógica para futuras 
extensiones del método, mostrando cómo el pensamiento numérico y el físico se complementan 
para alcanzar soluciones elegantes y rigurosas.

\end{document}
