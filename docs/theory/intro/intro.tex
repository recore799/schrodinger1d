% Introducción a la Mecánica Cuántica — versión concisa para la sección Recordar
\documentclass[11pt]{article}
\usepackage[utf8]{inputenc}
\usepackage[T1]{fontenc}
\usepackage[spanish]{babel}
\usepackage{amsmath,amssymb,bm,physics}
\usepackage{geometry}
\usepackage{hyperref}
\geometry{a4paper,margin=1in}
\title{Introducción a la mecánica cuántica}
\date{}
\begin{document}
\maketitle

\begin{abstract}
En este apartado recordamos los conceptos fundamentales de la mecánica cuántica: los postulados que definen su marco teórico y la interpretación de los observables físicos como operadores lineales en un espacio de Hilbert. Esta síntesis sirve como punto de partida para el estudio de modelos cuánticos canónicos y para el desarrollo de métodos numéricos como los empleados en este proyecto.
\end{abstract}

\section{Postulados básicos}
\label{sec:postulados}

\subsection{Postulado I: Estado cuántico}
El estado de un sistema físico se describe completamente mediante una función de onda (o vector de estado)
\[
\psi(\vb{r},t) \in \mathcal{H},
\]
donde \(\mathcal{H}\) es un espacio de Hilbert complejo.  
La probabilidad de encontrar la partícula en una región \(V\) del espacio está dada por
\[
P(V) = \int_V |\psi(\vb{r},t)|^2 \dd^3 r,
\]
y la normalización impone
\[
\int_{\mathbb{R}^3} |\psi|^2 \dd^3 r = 1.
\]

\subsection{Postulado II: Observables y operadores}
A cada magnitud física medible le corresponde un \textbf{operador hermítico} \(\hat{A}\) que actúa sobre los estados del sistema.  
Las posibles medidas de \(A\) son los \textbf{autovalores} de \(\hat{A}\), obtenidos de la ecuación
\[
\hat{A}\psi_a = a\psi_a,
\]
donde \(\psi_a\) son los autovectores ortogonales del operador.

\subsection{Postulado III: Medición y colapso}
Una medición de \(A\) sobre un sistema en el estado \(\psi\) produce con probabilidad
\[
P(a) = |\langle \psi_a | \psi \rangle|^2
\]
el valor \(a\), tras lo cual el estado del sistema colapsa en \(\psi_a\).

\subsection{Postulado IV: Evolución temporal}
La evolución temporal de un sistema aislado está gobernada por la ecuación de Schrödinger:
\[
i\hbar \pdv{\psi(\vb{r},t)}{t} = \hat{H}\psi(\vb{r},t),
\]
donde \(\hat{H}\) es el operador Hamiltoniano que representa la energía total del sistema.

\section{Operadores y valores esperados}
\label{sec:esperados}

Dado un observable \(\hat{A}\) y un estado normalizado \(\psi\), el \textbf{valor esperado} de \(A\) es
\[
\langle A \rangle = \langle \psi | \hat{A} | \psi \rangle
= \int \psi^*(\vb{r})\,\hat{A}\psi(\vb{r})\,\dd^3 r.
\]
Este valor representa el promedio de muchas mediciones idénticas sobre sistemas preparados en el mismo estado.

\subsection{Ejemplos comunes}
\begin{itemize}
\item Operador de posición: \(\hat{\vb{r}} = \vb{r}\)
\item Operador de momento: \(\hat{\vb{p}} = -i\hbar\nabla\)
\item Operador de energía (Hamiltoniano): \(\hat{H} = \frac{\hat{\vb{p}}^2}{2m} + V(\vb{r})\)
\end{itemize}

Estos operadores obedecen la relación de conmutación fundamental:
\[
[\hat{x}_i,\hat{p}_j] = i\hbar\,\delta_{ij}.
\]
De ella se deriva el principio de incertidumbre de Heisenberg:
\[
\Delta x_i\,\Delta p_i \geq \frac{\hbar}{2}.
\]

\section*{Resumen (Recordar)}
\begin{itemize}
\item Los estados cuánticos son vectores en un espacio de Hilbert y su módulo cuadrado da una probabilidad.
\item Los observables son operadores hermíticos con autovalores reales.
\item El valor esperado de un observable se calcula como \(\langle \psi|\hat{A}|\psi\rangle\).
\item La evolución temporal está regida por la ecuación de Schrödinger.
\end{itemize}

\end{document}
