\documentclass[11pt]{article}
\usepackage[utf8]{inputenc}
\usepackage[T1]{fontenc}
\usepackage{amsmath,amssymb,bm,physics}
\usepackage{geometry}
\usepackage{hyperref}
\geometry{a4paper,margin=1in}
\date{}
\title{Método de Hartree-Fock}

\begin{document}
\maketitle

\begin{abstract}
En este apartado recordamos la formulación matricial de las ecuaciones de Hartree–Fock para sistemas moleculares, conocidas como ecuaciones de Roothaan–Hall. Se presentan las expresiones de la matriz de densidad, de Fock y el procedimiento de autoconsistencia (SCF) utilizado para resolverlas. Este desarrollo constituye la base formal para la implementación computacional del método Hartree–Fock en una base finita de orbitales.
\end{abstract}

\section{Derivación de las ecuaciones de Hartree–Fock}
\label{sec:hf-derivation}

% Espacio reservado para la deducción variacional de las ecuaciones HF
\vspace{6cm}
% (Aquí incluirás la derivación de las ecuaciones de Hartree–Fock
% a partir del principio variacional y del determinante de Slater.)

\section{Ecuaciones de Roothaan–Hall}
\label{sec:roothaan}

Una vez que el espín se ha integrado, las ecuaciones de Hartree–Fock en forma espacial son
\[
f(1)\psi_i(1) = \epsilon_i\,\psi_i(1),
\]
donde \( f(1) \) es el operador de Fock.  
Al expandir los orbitales moleculares en una base finita de funciones \( \{\phi_\mu\} \),
\[
\psi_i = \sum_{\mu=1}^K C_{\mu i}\,\phi_\mu,
\]
las ecuaciones integradas de Hartree–Fock se transforman en la forma matricial de Roothaan:
\begin{equation}
\label{eq:roothaan}
\boxed{FC = SC\epsilon.}
\end{equation}

Aquí:
\begin{itemize}
\item \( F \) es la matriz de Fock con elementos
\[
F_{\mu\nu} = \int \dd{r_1}\,\phi_\mu^*(1)\,f(1)\,\phi_\nu(1),
\]
\item \( S \) es la matriz de traslape
\[
S_{\mu\nu} = \int \dd{r_1}\,\phi_\mu^*(1)\phi_\nu(1),
\]
\item \( C \) es la matriz de coeficientes de expansión,
\[
C = 
\begin{pmatrix}
C_{11} & C_{12} & \cdots & C_{1K} \\
C_{21} & C_{22} & \cdots & C_{2K} \\
\vdots & \vdots & \ddots & \vdots \\
C_{K1} & C_{K2} & \cdots & C_{KK}
\end{pmatrix},
\]
\item y \( \epsilon \) es la matriz diagonal de energías orbitales
\[
\epsilon = 
\begin{pmatrix}
\epsilon_1 & 0 & \cdots & 0 \\
0 & \epsilon_2 & \cdots & 0 \\
\vdots & & \ddots & \vdots \\
0 & 0 & \cdots & \epsilon_K
\end{pmatrix}.
\]
\end{itemize}

\section{Matriz de densidad}
\label{sec:density}

En el caso restringido (RHF) para sistemas de capa cerrada, la densidad electrónica es
\[
\rho(r) = 2\sum_{a}^{N/2}|\psi_a(r)|^2.
\]
Al sustituir la expansión en la base, se obtiene
\begin{align*}
\rho(r)
&= \sum_{\mu\nu}\!\qty[2\sum_a^{N/2}C_{\mu a}C_{\nu a}^*]\phi_\mu(r)\phi_\nu^*(r) \\
&= \sum_{\mu\nu}P_{\mu\nu}\phi_\mu(r)\phi_\nu^*(r),
\end{align*}
donde la matriz de densidad se define como
\begin{equation}
\label{eq:Pmat}
\boxed{P_{\mu\nu} = 2\sum_a^{N/2} C_{\mu a}C_{\nu a}^*.}
\end{equation}

\section{Matriz de Fock}
\label{sec:fock}

El operador de Fock puede escribirse como
\[
f(1) = h(1) + \sum_a^{N/2}[2J_a(1) - K_a(1)],
\]
y su representación matricial es
\begin{align*}
F_{\mu\nu}
&= \int \dd{r_1}\,\phi_\mu^*(1)h(1)\phi_\nu(1)
+ \sum_a^{N/2}\!\int \dd{r_1}\,\phi_\mu^*(1)[2J_a(1)-K_a(1)]\phi_\nu(1) \\
&= H_{\mu\nu}^{\text{core}} + \sum_a^{N/2}[2(\mu\nu|aa) - (\mu a|a\nu)].
\end{align*}

El término de energía de un electrón (Hamiltoniano del núcleo) se define como
\[
H_{\mu\nu}^{\text{core}} = \int \dd{r_1}\,\phi_\mu^*(1)h(1)\phi_\nu(1)
= T_{\mu\nu} + V_{\mu\nu}^{\text{nucl}}.
\]
Este término sólo se evalúa una vez en el procedimiento SCF.

Al expresar los orbitales moleculares en la base, el término de dos electrones se escribe en términos de la matriz de densidad:
\begin{align*}
F_{\mu\nu}
&= H_{\mu\nu}^{\text{core}} + \sum_{\lambda\sigma}
P_{\lambda\sigma}\qty[(\mu\nu|\sigma\lambda) - \frac{1}{2}(\mu\lambda|\sigma\nu)] \\
&= H_{\mu\nu}^{\text{core}} + G_{\mu\nu},
\end{align*}
donde
\begin{equation}
\label{eq:Gmat}
G_{\mu\nu} = \sum_{\lambda\sigma}
P_{\lambda\sigma}\qty[(\mu\nu|\sigma\lambda) - \frac{1}{2}(\mu\lambda|\sigma\nu)].
\end{equation}
Los integrales de dos electrones se definen como
\[
(\mu\nu|\lambda\sigma) = \int\!\!\dd{r_1}\dd{r_2}\,
\phi_\mu^*(1)\phi_\nu(1)\frac{1}{r_{12}}\phi_\lambda^*(2)\phi_\sigma(2).
\]

\section{Ortogonalización canónica}
\label{sec:orthogonalization}

Para obtener un conjunto ortonormal de funciones base, se busca una matriz \( X \) tal que
\[
X^\dagger S X = \mathbb{I}.
\]
La \textbf{ortogonalización canónica} usa la transformación
\begin{equation}
\label{eq:Xmat}
X = U s^{-1/2},
\end{equation}
donde \( U \) es la matriz unitaria que diagonaliza \( S \) y \( s \) es la matriz diagonal de sus autovalores:
\[
U^\dagger S U = s.
\]

Definimos un nuevo conjunto de coeficientes \( C' \) mediante
\begin{equation}
\label{eq:Ctrans}
C' = X^{-1}C, \qquad C = X C'.
\end{equation}
Al sustituir \( C \) en \eqref{eq:roothaan}, y multiplicar por \( X^\dagger \), se obtiene la forma transformada:
\begin{equation}
\label{eq:rooth-trans}
F' C' = C' \epsilon,
\end{equation}
con
\begin{equation}
\label{eq:Ftrans}
F' = X^\dagger F X.
\end{equation}
De esta manera, las ecuaciones de Roothaan–Hall se reducen a un problema de autovalores ordinario que puede resolverse por diagonalización.

\section{Procedimiento SCF}
\label{sec:scf}

El procedimiento de campo autoconsistente (\textit{Self-Consistent Field}, SCF) se resume así:
\begin{enumerate}
\item Especificar la molécula: coordenadas nucleares \(\{R_A\}\), números atómicos \(\{Z_A\}\), número de electrones \(N\) y base \(\{\phi_\mu\}\).
\item Calcular todos los integrales moleculares \(S_{\mu\nu}\), \(H_{\mu\nu}^{\text{core}}\) y \((\mu\nu|\lambda\sigma)\).
\item Diagonalizar \(S\) y obtener \(X\) según \eqref{eq:Xmat}.
\item Proponer una matriz de densidad inicial \(P\).
\item Calcular \(G_{\mu\nu}\) a partir de \eqref{eq:Gmat}.
\item Formar la matriz de Fock \(F = H^{\text{core}} + G\).
\item Calcular \(F' = X^\dagger F X\).
\item Diagonalizar \(F'\) para obtener \(C'\) y \(\epsilon\).
\item Calcular \(C = X C'\) y formar una nueva matriz de densidad \(P\) con \eqref{eq:Pmat}.
\item Verificar la convergencia. Si no converge, repetir desde el paso 5.
\end{enumerate}

\section*{Resumen (Recordar)}
\begin{itemize}
\item Las ecuaciones de Roothaan–Hall son la forma matricial del método de Hartree–Fock.
\item El problema \(FC=SC\epsilon\) se resuelve iterativamente mediante el procedimiento SCF.
\item La matriz de Fock depende de la densidad electrónica, y se actualiza hasta la autoconsistencia.
\end{itemize}

\end{document}
