\documentclass[11pt]{article}
\usepackage[utf8]{inputenc}
\usepackage[T1]{fontenc}
\usepackage{geometry}
\usepackage{amsmath,amssymb,physics}
\usepackage{listings}
\usepackage{hyperref}
\usepackage[spanish]{babel}
\geometry{a4paper,margin=1in}

\title{Documentación básica del método de Hartree–Fock restringido (RHF)}
\date{}

\begin{document}
\maketitle

\begin{abstract}
Este documento describe de manera concisa la implementación del método de Hartree–Fock restringido (RHF)
en el código desarrollado para moléculas diatómicas con una base mínima tipo \texttt{s}.
Se presentan los componentes principales del algoritmo, la estructura de módulos para el cálculo de integrales,
y el procedimiento de autoconsistencia (SCF). El objetivo es proporcionar una referencia breve
que documente las decisiones de diseño sin entrar en los desarrollos matemáticos de las integrales gaussianas.
\end{abstract}

\section{Introducción}
El método de Hartree–Fock (HF) busca obtener una aproximación determinantal a la función de onda
de un sistema electrónico, minimizando la energía total bajo la restricción de ortonormalidad de los orbitales.
La versión restringida (RHF) asume que los electrones de espín $\alpha$ y $\beta$
ocupan el mismo conjunto de orbitales espaciales, simplificando el cálculo.

En esta implementación, el procedimiento se desarrolla para una base mínima de funciones tipo Gaussiana,
suficiente para moléculas diatómicas como \(\mathrm{H_2}\) o \(\mathrm{HeH^+}\).

\section{Estructura del código}
El cálculo principal se realiza en el módulo \texttt{rhf\_s\_old.py}, apoyado por el módulo \texttt{s\_integrals.py},
que contiene las rutinas para el cálculo de los integrales de una base \texttt{s}-tipo.

\subsection{Cálculo de integrales atómicas}
El módulo \texttt{s\_integrals.py} construye los arreglos de integrales requeridos para el procedimiento SCF:

\begin{itemize}
\item \textbf{Matriz de traslape} \( S_{\mu\nu} = \langle \phi_\mu | \phi_\nu \rangle \)
\item \textbf{Matriz cinética} \( T_{\mu\nu} = \langle \phi_\mu | -\frac{1}{2}\nabla^2 | \phi_\nu \rangle \)
\item \textbf{Atracción nuclear} \( V_{\mu\nu} = \langle \phi_\mu | \sum_A \frac{-Z_A}{r_A} | \phi_\nu \rangle \)
\item \textbf{Integrales bielectrónicas} \((\mu\nu|\lambda\sigma)\)
\end{itemize}

Cada integral se calcula como suma de contribuciones entre primitivas gaussianas de tipo \((\alpha, d)\),
y las integrales bielectrónicas se almacenan en forma \emph{sparse} mediante un diccionario indexado
por claves canónicas que respetan las simetrías de permutación:

\[
(\mu\nu|\lambda\sigma) = (\nu\mu|\lambda\sigma) = (\mu\nu|\sigma\lambda) = \ldots
\]

\subsection{Construcción del Hamiltoniano}
Se define el Hamiltoniano de núcleo fijo:
\[
H_{\mu\nu}^{\text{core}} = T_{\mu\nu} + V_{\mu\nu}.
\]
Este operador se usa para obtener la energía de referencia (guess inicial)
y la energía electrónica total:
\[
E_{\text{elec}} = \sum_{\mu\nu} P_{\mu\nu}(H_{\mu\nu}^{\text{core}} + \tfrac{1}{2}F_{\mu\nu}),
\]
donde \(P_{\mu\nu}\) es la matriz de densidad.

\section{Algoritmo SCF (RHF)}
El procedimiento de autoconsistencia (SCF) implementado sigue los pasos clásicos:

\begin{enumerate}
\item \textbf{Inicialización de matrices:} se calculan \(S, T, V, H_{\text{core}}\)
y el diccionario de integrales bielectrónicas \texttt{eri\_dict}.
\item \textbf{Ortonormalización simétrica:} se obtiene \(X = S^{-1/2}\) mediante
\texttt{fractional\_matrix\_power(S, -0.5)}.
\item \textbf{Conjetura inicial:} matriz de densidad \(P=0\).
\item \textbf{Bucle SCF:}
  \begin{enumerate}
  \item Construcción del operador de Fock:
  \[
  F_{\mu\nu} = H_{\mu\nu}^{\text{core}} + \sum_{\lambda\sigma} P_{\lambda\sigma}
  \left[ (\mu\nu|\lambda\sigma) - \tfrac{1}{2}(\mu\lambda|\nu\sigma) \right].
  \]
  \item Transformación a la base ortonormal: \( F' = X^T F X \).
  \item Diagonalización de \(F'\) para obtener los orbitales moleculares \(C'\) y energías \(\epsilon\).
  \item Retrotransformación: \(C = X C'\).
  \item Construcción de la nueva densidad:
  \[
  P_{\mu\nu} = 2 \sum_{i}^{\text{occ}} C_{\mu i} C_{\nu i}.
  \]
  \item Cálculo de la energía total y prueba de convergencia (\(\Delta P_{\text{rms}} < 10^{-6}\)).
  \end{enumerate}
\end{enumerate}

El ciclo termina cuando la matriz de densidad converge.  
La energía total final se obtiene como:
\[
E_{\text{total}} = E_{\text{elec}} + E_{\text{nuc}}.
\]

\section{Módulos auxiliares}
El código incluye utilidades adicionales:
\begin{itemize}
\item \texttt{compute\_nuclear\_repulsion\_energy}: calcula la energía de repulsión nuclear generalizada.
\item \texttt{primitives\_to\_ao\_list}, \texttt{ao\_centers\_from\_pos}, \texttt{atom\_of\_mu\_from\_primitives}: funciones de mapeo entre primitivas, átomos y orbitales.
\item \texttt{print\_final\_results}: genera un resumen final de la energía y los orbitales convergidos.
\end{itemize}

\section{Notas sobre implementación}
\begin{itemize}
\item Las integrales están programadas para una base \texttt{s}-tipo; 
la extensión a \texttt{p} u orbitales de mayor momento angular requiere nuevos tensores de integrales.
\item La estructura \texttt{eri\_dict} reduce significativamente el costo en memoria respecto a un tensor \(N^4\).
\item El algoritmo usa \textbf{ortonormalización simétrica}, que es estable incluso cuando \(S\) es mal condicionado.
\end{itemize}

\section*{Conclusión}
Este módulo constituye una implementación autocontenida y didáctica del método de Hartree–Fock restringido.
Aun en su forma básica, permite explorar los principios del método SCF, el papel de las integrales atómicas,
y la construcción del operador de Fock. Futuras extensiones pueden incluir:
uso de bases extendidas (STO-3G, 6-31G), incorporación de integrales \(p\)-tipo,
y el desarrollo de módulos post-Hartree–Fock.

\end{document}
