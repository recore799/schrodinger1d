\documentclass[12pt,a4paper]{article}

\usepackage[utf8]{inputenc}
\usepackage[T1]{fontenc}
\usepackage[spanish]{babel}
\usepackage{amsmath, amssymb}
\usepackage{geometry}
\usepackage{physics}
\usepackage{bm}
\usepackage{hyperref}
\usepackage{graphicx}
\geometry{margin=1in}

\title{Derivación Variacional de las Ecuaciones de Hartree--Fock}
\author{Basado en \textit{Modern Quantum Chemistry} de Szabo y Ostlund}
\date{}

\begin{document}
\maketitle

\section{Introducción}

El método de Hartree--Fock busca encontrar la mejor aproximación para la función de onda de un sistema de \( N \) electrones, suponiendo que esta puede escribirse como un único determinante de Slater,
\[
\ket{\Psi_0} = \frac{1}{\sqrt{N!}} \det[\chi_1(1)\,\chi_2(2)\,\dots\,\chi_N(N)],
\]
donde cada \(\chi_a\) es un espín–orbital ortonormal.

El objetivo es minimizar el valor esperado de la energía,
\[
E_0 = \mel{\Psi_0}{\hat{H}}{\Psi_0},
\]
que es un funcional de los espín–orbitales \(\{\chi_a\}\).  
La minimización debe realizarse bajo la restricción de ortonormalidad:
\[
[a|b] \equiv \int \chi_a^*(x_1)\chi_b(x_1)\,dx_1 = \delta_{ab}.
\]

\section{El funcional a minimizar}

La energía asociada a un determinante de Slater se puede expresar como
\[
E_0[\{\chi_a\}] = \sum_{a=1}^{N} [a|h|a] 
+ \frac{1}{2} \sum_{a,b=1}^{N} \left( [aa|bb] - [ab|ba] \right),
\]
donde \( [a|h|a] \) representa las integrales de un electrón (energía cinética más atracción nuclear), y los términos \( [aa|bb] \) y \( [ab|ba] \) son las integrales de dos electrones de Coulomb e intercambio, respectivamente.

Para incluir las restricciones de ortonormalidad introducimos multiplicadores de Lagrange \(\epsilon_{ba}\) y definimos el funcional de Lagrange:
\[
\mathcal{L}[\{\chi_a\}] = 
E_0[\{\chi_a\}] 
- \sum_{a,b=1}^{N} \epsilon_{ba} \left( [a|b] - \delta_{ab} \right).
\]
Los multiplicadores \(\epsilon_{ba}\) forman una matriz Hermítica, \(\epsilon_{ba} = \epsilon_{ab}^*\), garantizando que \(\mathcal{L}\) sea real.

\section{Condición variacional}

El mínimo del funcional se obtiene exigiendo que su primera variación sea nula:
\[
\delta \mathcal{L} = 0.
\]
Esto implica que cualquier cambio infinitesimal en los espín–orbitales,
\(\chi_a \to \chi_a + \delta\chi_a\),
no altera el valor de \(\mathcal{L}\) en el punto estacionario.

Después de una variación funcional y de agrupar términos, se obtiene:
\[
\delta\mathcal{L} = 
\sum_{a=1}^{N} 
\int \delta\chi_a^*(1)
\left[
h(1)\chi_a(1)
+ \sum_{b=1}^{N} (J_b(1) - K_b(1))\chi_a(1)
- \sum_{b=1}^{N}\epsilon_{ba}\chi_b(1)
\right] dx_1
+ \text{c.c.}
\]
Como las variaciones \(\delta\chi_a^*\) son arbitrarias, el término entre corchetes debe anularse:
\[
\left[
h(1) + \sum_{b=1}^{N} (J_b(1) - K_b(1))
\right]\chi_a(1)
= \sum_{b=1}^{N}\epsilon_{ba}\chi_b(1).
\]
Definimos el operador de Fock,
\[
f(1) = h(1) + \sum_{b=1}^{N} (J_b(1) - K_b(1)),
\]
y las ecuaciones anteriores se pueden escribir de manera compacta como
\[
f \ket{\chi_a} = \sum_{b=1}^{N} \epsilon_{ba} \ket{\chi_b}.
\]

\section{Invarianza unitaria y forma canónica}

Un determinante de Slater es invariante (salvo un factor de fase) bajo una transformación unitaria de los espín–orbitales:
\[
\chi'_a = \sum_b \chi_b U_{ba},
\quad
U^\dagger U = I.
\]
El nuevo determinante \(\ket{\Psi'_0}\) difiere del original solo por un factor de fase, por lo que representa el mismo estado físico.

El operador de Fock es invariante bajo esta transformación, \(f' = f\), pero la matriz de multiplicadores se transforma como
\[
\epsilon' = U^\dagger \epsilon U.
\]
Como \(\epsilon\) es Hermítica, siempre existe una transformación unitaria \(U\) que la diagonaliza:
\[
\epsilon'_{ab} = \epsilon'_a \delta_{ab}.
\]
Con esta elección, las ecuaciones de Hartree–Fock toman la forma de eigenvalores:
\[
f \ket{\chi'_a} = \epsilon'_a \ket{\chi'_a}.
\]
Omitiendo las primas para simplificar la notación:
\[
f \ket{\chi_a} = \epsilon_a \ket{\chi_a}.
\]

\section{Conclusión}

Las ecuaciones canónicas de Hartree--Fock constituyen un sistema no lineal, ya que el operador de Fock depende de los propios orbitales que se buscan.  
Por lo tanto, deben resolverse de manera iterativa mediante un procedimiento de autoconsistencia (SCF, \textit{Self-Consistent Field}).  
Los eigenvalores \(\epsilon_a\) se interpretan como las energías orbitales, y los orbitales \(\chi_a\) correspondientes son los \textbf{espín–orbitales canónicos}.

\end{document}
