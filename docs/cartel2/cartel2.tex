\documentclass[final]{beamer}
\usepackage[orientation=portrait,size=a0,scale=1.2]{beamerposter}
\usepackage[utf8]{inputenc}
\usepackage[T1]{fontenc}
\usepackage{lmodern}
\usepackage[spanish]{babel}
\usepackage{graphicx} % Para incluir imágenes
\usepackage{amsmath}  % Para ecuaciones matemáticas
\usepackage{multicol} % Para controlar columnas de texto si fuera necesario dentro de un bloque
\usepackage{caption}  % Para controlar las leyendas de las figuras (captionof)

% --- Configuración del Tema y Colores (Opcional, puedes cambiarlo) ---
% Para pósters, a menudo se usa un tema minimalista o se crea uno propio.
% Aquí usaremos un tema base y ajustaremos.
\usetheme{Madrid} % Puedes elegir otros temas como Berlin, Warsaw, etc.
\usecolortheme{default} % Tema de color por defecto

% Ajustes para el tamaño de la fuente si el póster es grande
% \documentclass[final, bigger]{beamer} % o similar, dependiendo de la distribución de Beamer para pósters
% Para pósters, a veces se usa una plantilla específica como 'beamerthemeposter' o se ajusta el tamaño de la fuente.
% Como estamos usando beamer directamente, haremos un "póster" con frames.
% Si buscas un póster de gran formato, considera usar la clase 'a0poster' o paquetes especializados.
% Para este ejercicio, asumiremos que es una presentación tipo póster con diapositivas grandes.

% --- Información del Cartel ---
\title{Título de tu Cartel Científico}
\author{Tu Nombre / Nombres de los Autores}
\institute{Tu Institución o Afiliación}
\date{\today}

\begin{document}

% --- Frame Principal del Póster (una sola "diapositiva" grande para el póster) ---
\begin{frame}[t] % 't' alinea el contenido en la parte superior

\begin{center}
    \textbf{\Huge \inserttitle} \\ % Título grande
    \vspace{0.5cm}
    \Large \insertauthor \\ % Autor
    \small \insertinstitute \\ % Institución
    \vspace{0.8cm}
\end{center}

% Estructura de Columnas para simular tu diagrama
\begin{columns}[T] % 'T' alinea las columnas por la parte superior

    % --- Columna Izquierda (aproximadamente 45% del ancho) ---
    \begin{column}{0.48\textwidth} % Ancho de la columna izquierda

        % Fila 1: Intro
        \begin{block}{Introducción}
            Aquí va el texto de tu introducción. Explica el contexto, la motivación
            y los objetivos principales de tu investigación. Sé conciso y directo.
        \end{block}

        % Fila 2: Numeral / Marco Teórico
        \begin{block}{Numeral / Marco Teórico}
            Desarrolla los conceptos fundamentales o el marco teórico que sustenta tu trabajo.
            Puedes incluir ecuaciones relevantes, por ejemplo:
            $$ E = mc^2 $$
            O descripciones de modelos.
        \end{block}

        % Fila 3: Tres Imágenes Apiladas (Una a la izquierda, dos apiladas a la derecha)
        % Para simular la cuadrícula "Im | Im", "Im"
        \begin{block}{Imágenes y Conceptos}
            \begin{figure}
                \centering
                % Primera imagen (ocupando el ancho del bloque)
                \includegraphics[width=0.9\linewidth]{example-image-a}
                \captionof{figure}{Descripción de la primera imagen.}
            \end{figure}
            \vspace{0.5cm}
            \begin{figure}
                \centering
                % Segunda imagen (ocupando el ancho del bloque)
                \includegraphics[width=0.9\linewidth]{example-image-b}
                \captionof{figure}{Descripción de la segunda imagen.}
            \end{figure}
            \vspace{0.5cm}
            \begin{figure}
                \centering
                % Tercera imagen (ocupando el ancho del bloque)
                \includegraphics[width=0.9\linewidth]{example-image-c}
                \captionof{figure}{Descripción de la tercera imagen.}
            \end{figure}
            % Si quieres que sean dos imágenes lado a lado y una abajo,
            % tendrías que anidar 'columns' dentro del block, pero para simplicidad
            % las apilamos para que quepan bien en el espacio asignado.
        \end{block}

        % Fila 4: Metodología / Didáctica
        \begin{block}{Metodología y Didáctica}
            Describe los métodos utilizados en tu investigación. Si es aplicable,
            detalla el enfoque didáctico o las herramientas pedagógicas empleadas.
            \begin{itemize}
                \item Paso 1 del método
                \item Paso 2 del método
                \item Consideraciones didácticas si las hay.
            \end{itemize}
        \end{block}

    \end{column}

    % --- Columna Derecha (aproximadamente 45% del ancho) ---
    \begin{column}{0.48\textwidth} % Ancho de la columna derecha

        % Fila 1: Imagen Principal Derecha
        \begin{block}{Imagen de Contexto}
            \begin{figure}
                \centering
                \includegraphics[width=\linewidth]{example-image-a} % Reemplaza con tu imagen
                \captionof{figure}{Imagen principal relacionada con el título o la introducción.}
            \end{figure}
        \end{block}

        % Fila 2: Dos Imágenes Pequeñas
        \begin{block}{Ilustraciones Clave}
            \begin{figure}
                \centering
                \includegraphics[width=0.48\linewidth]{example-image-b} % Primera imagen pequeña
                \includegraphics[width=0.48\linewidth]{example-image-c} % Segunda imagen pequeña
                \captionof{figure}{Imágenes que complementan el marco teórico o la metodología.}
            \end{figure}
        \end{block}

        % Fila 3: Sección Hartree-Fock
        \begin{block}{Método Hartree-Fock}
            Aquí puedes desarrollar la sección específica sobre el método Hartree-Fock.
            Explica sus principios, aplicaciones y resultados relevantes en tu estudio.
            \begin{itemize}
                \item Descripción teórica.
                \item Ecuaciones clave (e.g., ecuaciones de Hartree-Fock).
                $$ F \phi_i = \epsilon_i \phi_i $$
                \item Resultados obtenidos usando este método.
            \end{itemize}
        \end{block}

        % Fila 4: Entrevista
        \begin{block}{Resultados de Entrevista / Análisis}
            Si realizaste entrevistas, aquí presenta los hallazgos clave, citas relevantes
            o un análisis cualitativo de los datos obtenidos.
            \begin{itemize}
                \item Puntos principales de las entrevistas.
                \item Análisis de los datos.
                \item Implicaciones.
            \end{itemize}
        \end{block}

        % Fila 5: Conclusión y QR
        \begin{block}{Conclusión y Contacto}
            Resume los hallazgos más importantes de tu cartel y sus implicaciones.
            Asegúrate de responder a los objetivos planteados en la introducción.
            \begin{itemize}
                \item Resumen de los resultados.
                \item Contribuciones principales.
                \item Trabajo futuro.
            \end{itemize}
            \vspace{0.5cm}
            \centering
            % Aquí puedes poner una imagen de un código QR
            \includegraphics[width=0.3\linewidth]{example-image-1x1} % Reemplaza con tu código QR
            \captionof{figure}{Escanea para más información o contacto.}
        \end{block}

    \end{column}

\end{columns}

\end{frame}

\end{document}
